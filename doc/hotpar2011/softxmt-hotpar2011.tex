%% move stuff to defs file
\newcommand{\todo}[1]{{\bfseries [[#1]]}}
%% To disable, just uncomment this line
%% \renewcommand{\todo}[1]{\relax}

\documentclass{acm_proc_article-sp}
\usepackage{url}
\begin{document}

\title{Can We Crunch the Social Graph on the Cheap?}

\numberofauthors{7}
\author{
(authors)
}

\maketitle
\begin{abstract}

  Large irregular applications such as sparse, low diameter graph problems often
  perform badly on commodity clusters. These application make many
  non-sequential accesses to global data structures with low degrees of
  locality, interacting badly with commodity processors' memory
  systems, which are optimized for local, sequential memory
  accesses. Special-purpose hardware optimized for these irregular
  applications has been built; the Cray XMT is one such machine. It
  uses multithreading to tolerate the latency of low-locality memory
  accesses, and while it exhibits excellent performance and energy
  efficiency for irregular applications, it performs badly on standard
  sequential applications \todo{Brandon: should this be more specific or worded differently? Its true sequential perform badly since switching between many threads, but also does XMT also perform  poorly (relative to alternatives) on highly parallel but regular apps with lots of temporal and spatial locality?}. We believe we can have the best of both
  worlds: we argue that modern commodity processors have many of the
  features necessary \todo{Brandon: we will have to be able to list these as features in the paper or use a different phrase; my feeling is that these are aspects of the system which are not optimized for this but are sufficient to suggest potential for enough memory concurrency} to give good performance on irregular problems,
  and that a latency-tolerant runtime designed to exploit these
  features can enable speedup for irregular applications on commodity
  processors. In this paper, we describe the key features of such a
  runtime system, and what a cluster built using our runtime might
  look like. Using a preliminary single-node implementation of our
  runtime, we obtained speedups on simple list chasing benchmarks. Our results show promise for a multi-node implementation.

\end{abstract}

\section{Introduction}


Graphs have proven to be a useful abstraction for solving real-world
problems in social and biological network analysis. But many graphs in
social and biological network analysis are sparse, with low diameter
and a degree distribution that follows a power law. These properties
make it difficult to take advantage of the locality assumptions built
into the memory hierarchy of modern processors, so that even simple
traversals are dominated by memory latency.

And these graphs can be large---too large for a single node \todo{may want to avoid saying 'node' since graph 'node' is also used in this paragraph}. In January 2011, Facebook had 500 million active users with an average of
130 friends each \cite{Facebook:2011p91}. A natural approach would be
to try to partition the graph across the nodes in a cluster. But with
low diameter graphs, a traversal starting at any node is likely to
quickly leave its partition and travel over the network, incurring
high latency costs. And power law graphs have a small number of nodes
that generate significantly more work than other nodes, leading to
computational imbalance.

% Facebook's friend suggestion
% algorithm in 2010 ran on a 40-node cluster with 72GB memory per
% node. Even with that capacity, new suggestions could only be computed
% every two days \cite{Backstrom:2010p90}.


Multithreading is a technique that has been used successfully to
implement efficient computations for these graphs. The Cray XMT is an
example of such a solution: it tries to solve the memory latency
problem through concurrency rather than caching. Each XMT processor
supports 128 hardware contexts and $\sim$1000 outstanding memory
operations and is able to switch contexts every cycle. But this
approach has a limitation: each context is significantly slower than a
modern commodity processor, yielding bad single-threaded
performance. While the XMT is fast at sparse, low-degree graph
problems, it is slow at other scientific codes that can take advantage
of commodity processors' memory hierarchies. Thus, few XMTs have been
built.

We want the best of both worlds---we want to build a system that has
good performance on both these low-locality graph codes as well as on
common scientific codes. Our approach is inspired by the XMT, but
implemented in software on commodity microprocessors. The core of our
approach is a lightweight coroutine library designed to overlap
long-latency memory prefetches with other contexts' computation. 

\todo{something like: project still in early stages. this document lays out
  problems we think a system like this needs to solve, and describes
  some implementation strategies we want to explore. we have a basic
  initial implementation with some experiments that suggest there is
  promise here.}

\todo{do we want to mention hardware at this point? If so, we should
  perhaps call it a hybrid approach.}

\section{Programming model}

While the ideal programming model is one that facilitates easy
expression in the problem domain as well as efficient execution on the
available hardware, our programming model today is motivated by only
efficient execution: it exists as a rudimentary library for concept
demonstration, not as a language suitable for development.  The
programming model we use today is therefore only partially formed. It
exists as an augmentation of existing threaded programming models such
as pthreads or openmp that are readily available on most computer
systems.

Though not strictly necessary, for ease of exposition we imagine a
one-to-one binding between threads and cores.  Our model manifests
within any thread independently of the others as a ``fray'' \todo{for lack
of better term }.  Within the thread, a fray instance is the
instantiation of a set of coroutines and a scheduler.  Instantiation
is at user-level and the operating system sees the entire fray as as a
single user-level thread.

\todo{ describe calls that perform thread \& scheduler instantiation; and schedule activation }

Each coroutine executes non-preemptively, utilizing the core to which
its ``parent'' thread has been assigned by the operating system until it
reaches a yield point.  Typically, the coroutine yields after
initiating a prefetch operation on data that the programmer or
compiler anticipates would probably not complete before the data is
referenced by a subsequent load operation.  In this way, the core
continues to execute instructions for other coroutines when otherwise
it would likely stall.  The programming model thus tolerates latency.

\todo{ describe calls to prefetch-and-yield }

When a thread frays, for the sake of expediency, a fixed portion of
its stack is divided amongst the coroutines.  Given a finite stack,
each coroutine is constrained to execute a call tree of depth known to
be finite at compilation time: recursion is not yet permitted.

\todo{ discussion of coroutines \& synchronization } Were a coroutine to
block on a synchronization variable shared with other threads, the
entire fray would suspend execution.  This can lead to deadlock when,
for example, one coroutine waits to consume from another thread that
is waiting to consume what only another coroutine in this first fray
can produce.  Instead, coroutines must yield on failed synchronization
events, spinning rather than blocking, where were they bona fide
threads, blocking might be more efficient.

\todo{ say something about ``synchronization'' within a fray }

Coroutines completing their work yield without adding themselves to
the scheduling queue.  The last coroutine to exit in this way returns
as the main thread, as in the common  fork-join model of parallelism.


\section{Approach}

Our goal is to enable high throughput on graphs with two challenging
properties. First, the graphs' combination of sparseness and low
degree means there is little locality to be exploited. Second, the
graphs are larger than a single node's memory and do not partition
well \todo{explain?}. We must spread the graph across multiple nodes
and send memory operations over a network. If memory requests are
uniformly distributed over the working set, a given node will make as
many accesses to a distant node's memory as it does to the local node.

We would like to run as fast as current hardware will let us. Pin
bandwidth is the tightest resource: Intel's QuickPath Interconnect
(QPI) allows approximately 100 million requests per second
\todo{cite}, and recent interconnect products allow similar request
rates, with a round-trip latency of approximately 3$\mu$s
\todo{cite}. Little's law tells us then that we must support on the
order of 300 concurrent memory operations, and thus coroutines, to
cover this latency.

These requirements drive the design of our system. There are three key
components.

\subsection{Low-latency context switch}

There are three main concerns in designing the coroutine
mechanism. First, contexts must be small, so many coroutines can be
active without monopolizing the cache. Second, context switches must
be fast, so that overhead does not swamp actual work. Third, when a
coroutine yields, we must choose a good coroutine to run next.

We minimize context size by treating context switches as function
calls, as in \cite{charm}. This allows us to save and restore the
minimum number of registers allowed by the ABI, and depend on the
compiler to save and restore any other registers in use. We minimize
context switch time both by keeping contexts small and by doing all
switching and scheduling in user space, as in \todo{cite other green
  threads packages}.

The choice of which coroutine to run next presents us with a
tradeoff. The simplest scheduler chooses coroutines to execute in
round-robin fashion. But this may lead to performance problems: a
coroutine might be scheduled before the event which caused it to yield
has completed, stalling the pipeline and keeping other coroutines from
progressing. A coroutine might also be scheduled long after the event
which caused it to yield has completed; while this wouldn't block
other coroutines from proceeding, it would add unnecessary latency to
the yielding coroutine. A more intelligent scheduler might allow
coroutines to execute only when they were ready, but this could
increase the overhead of context switching. Our initial implementation
takes the simple approach, but more investigation is warranted.

\subsection{Accessing Global Memory}

% Our goal is to support many concurrent accesses to a global address
% space spread across nodes in a cluster. We can draw on hardware
% support developed for the Partitioned Global Address Space family of
% languages to accomplish this.

% We draw on libraries and network support 
% goal. Languages in the Partitioned Global Address Space family such as
% X10 and Chapel have similar goals, 


% It is also useful to access this low-locality data without disturbing
% the thread contexts and local storage in the cache. Current x86
% processors support non-temporal loads and stores. These memory
% operations are quite relevant for our problem: they are designed to
% move data without polluting the cache. Using these instructions works
% better than mapping the low-locality data as uncached, since that
% restricts the processors' ability to reorder operations.



Following the lead of the PGAS family of languages, we will use
the RDMA functionality provided by modern Infiniband networks to
access our global memory. \todo{rethink?}

These networks allow us to read and write memory directly from remote
nodes, but they do so through a low-level interface designed to be
hidden behind a library. Loads and stores to global memory must be
replaced with calls to a library. This library causes the network
interface to fetch the remote data and place it in a temporary
buffer. When the requesting coroutine is scheduled again, data is
returned from the buffer and execution proceeds.


In order to minimize memory access overhead, the memory operation
library calls do not interact with the network interface
directly. Instead, they queue their operations to another thread, the
{\em memory manager thread}, running on another core in the CPU. This thread
translates the global address into a network destination and issues
the network operation, checks for completion, and manages the returned
data buffers.

But we have to be careful---directing memory requests from multiple
compute threads to a single memory manager thread may reduce memory
concurrency. And we've increased the overhead of a single memory
operation---what in the single node case was a simple read, is now
multiple queuing operations between threads, and multiple reads and
writes over QPI to the network interface. Adding more memory manager
threads may enable more memory concurrency, but the overhead is still high.

Instead, we propose hardware acceleration with a {\em memory manager
  FPGA}, a coprocessor FPGA inserted into a Xeon socket so it can
participate in the node's coherence domain. This coprocessor does no
computation---it would only manage the global memory requests of the
main CPU, performing the function of the memory manager thread
described earlier.

By participating in the coherence domain, the memory manager FPGA can
act as a proxy for the rest of global memory. It advertises the
address space of the rest of the system, and translates local loads
and stores to addresses on remote nodes into the network operations
required.

Unfortunately, current processors do not provide sufficient hardware
resources for us to reach our concurrency target using this
approach. Each core in Intel's X5650 Xeon supports only 10 outstanding
misses from its private L2; the entire chip is limited to 32 L2 misses
to local memory and 12 L2 misses for memory reached over the QPI
bus. A simple translation of a local load to a network access would
tie up one of these slots for the entire multi-microsecond latency of
the request. We must bypass these structures to allow the amount of
concurrency we need; the CPU needs to direct the memory manager FPGA
to do work on its behalf without tying up its own resources.

To accomplish this, we encode commands in the high-order bits of
remote memory addresses. The memory manager FPGA advertises each
remote address twice. The first address is used for blocking reads and
writes---any access incurs the full latency of the remote
access. \todo{verify writes block?} The second address is used for
prefetches and asynchronous writes. These addresses differ by one
high-order bit, easily set with a mask in a macro. A read from the
prefetch address starts the network transaction, but immediately
returns a dummy value. When the value arrives, it is stored in a
temporary buffer; when the requesting coroutine is scheduled again, it
executes the blocking read, and the memory manager immediately returns
the value from the buffer. \todo{this isn't fully baked---do we want
  to include this sort of detail?}

\subsection{Synchronization}

In systems with hundreds of threads per node, lightweight
synchronization is important. While we have not implemented any
synchronization primitives yet, full-empty bits are a natural choice
given our programming model. Two questions then arise: at what
granularity should full-empty bits work, and what component should do
the synchronization?

We expect most data accesses to be to 64-bit words. It is possible to
support full-empty bit synchronization on arbitrary words by
allocating additional storage for the full-empty bits, and translating
each atomic full-empty operation into a sequence of memory operations
that modify the data and full-empty bit atomically. 

One potential optimization is to limit full-empty synchronization to
pointers to aligned 64-bit words. This leaves the bottom three bits
free to store the full-empty bits. These bits would be masked out when
the pointer was returned to the user.

These techniques can be implemented in software; \cite{qthreads} is
one example. Alternatively, atomic operations could be delegated to a
memory manager FPGA as described in the previous section. The address
command encoding would be extended to support the various full-empty
operations.

It might seem natural to use the FPGA as a memory controller as
well, storing all full-empty-bit-capable memory in DRAMs attached to
the FPGA. While this is conceptually simple, it does make the
programming model and the design more complicated. Another possibility
would be to store only the full-empty bits in memories attached to the
FPGA, and leave data in memory attached to the CPU. We believe that
starting simple is best: both data and full-empty bits should be
stored in the CPU's memory. \todo{discuss QPI bandwidth differences}



\section{Current Implementation and Results}

\todo{Begin eval insert}
\section{Evaluation}
To evaluate the impact flat combining has on the global data structures, we ran a series of experiments to test the raw performance of the data structures themselves under different workloads, and further measured their impact on performance of two simple graph benchmarks.

Experiments were run on a cluster of AMD Interlagos processors. Nodes have 32 2.1-GHz cores in two sockets, 64GB of memory, and 40Gb Mellanox ConnectX-2 InfiniBand network cards, connected via a QLogic InfiniBand switch.

\subsection{Data Structure Throughput}
First we measured the raw performance of the global data structures on extremely simple throughput workloads.

% \begin{figure*}[t]
%   \centering
\begin{figure}[t] %{0.5\textwidth}
  \centering
  \includegraphics[width=0.5\textwidth]{data/plots/vector_perf.pdf}
  \caption{\emph{GlobalStack and GlobalQueue.}
    Results are shown on a log scale for a throughput workload performing 256 million operations with 2048 workers per core and 16 cores per node. Flat combining improves throughput by at least an order of magnitude and allows performance to scale. Matching pushes and pops enables the stack to perform even better on a mixed workload.
  }
  \label{fig:vector}
\end{figure}
  % \hspace{0.05\textwidth}
  % % ~ %add desired spacing between images, e. g. ~, \quad, \qquad etc.
  % %(or a blank line to force the subfigure onto a new line)
  % \begin{subfigure}[b]{0.45\textwidth}
  % \centering
  % \includegraphics[width=\textwidth]{data/plots/queue_perf.pdf}
  % \caption{\emph{GlobalQueue.} Same parameters as stack performance results. The queue is unable to do matching locally, but benefits from reducing the amount of synchronization that must globally serialize. The mixed workload performs worse because the current implementation serializes combined enqueue and dequeue operations.}
  % \label{fig:queue}
  % \end{subfigure}
  % \hspace{0.05\textwidth}
  
  %  
\begin{figure}[t]
  \centering
  \includegraphics[width=0.5\textwidth]{data/plots/hash_perf.pdf}
  \caption{\emph{GlobalHashSet and GlobalHashMap.}
    Results are shown for a throughput workload inserting and looking up 256 million random keys in the range 0-1024 into a global hash with 1024 cells, with 2048 workers per core and 16 cores per node.
    Performance without combining scales out to 32 nodes because synchronization happens at each hash cell, but drops off as the number of destinations increases. Due to eliminating duplicate inserts and lookups, the combining version is able to continue to scale.}
    \TODO{if time: add 'delete' operation, too (show that it doesn't affect correctness or performance)}
  \label{fig:hash_perf}
\end{figure}
  % %
  % \hspace{0.05\textwidth}
  % %
  % \begin{subfigure}[b]{0.45\textwidth}
  % \centering
  % \includegraphics[width=\textwidth]{data/plots/hashmap_perf.pdf}
  % \caption{\emph{GlobalHashMap.} Same workload as for the Set, but random integers as the values in the map. Performance matches that of the Set.} \TODO{try increasing size of value "payload", currently is just a tiny int64.}
  % \label{fig:hashmap}
  % \end{subfigure}%
  %
%   \caption{
%     \emph{Raw performance of global data structures on simple throughput workloads.}
%     Results shown for workloads with varying mixtures of writing and reading operations. All experiments were performed with 16 cores per node, and a suitable set of Grappa runtime parameters fixed for a given plot.
%     \TODO{either all with error bars or none?}
%     \TODO{combine stack/queue \& set/map plots, no point in saying the same things twice.}
%   }\label{fig:datastructs}
% \end{figure*}

\paragraph{Queue and Stack}
The GlobalQueue and GlobalStack have very similar implementations in terms of how they are synchronized. Even with Grappa's aggregation, without combining, both the stack and queue completely fail to scale because of serialization of all workers' updates on the master core.
With combining, both scale well out to 64 nodes.
On the mixed workload, the stack is able to do matching locally, allowing it to reduce the amount of communication drastically, greatly improving its performance.

The queue also benefits from reduced synchronization and batching operations, and its all-push workload performs identically to the stack's.
However, the queue is unable to do matching locally, and in fact, the mixed workload performs worse because the current implementation serializes combined enqueue and dequeue operations. This restriction could be lifted with more careful synchronization at the master core allowing enqueues and dequeues to proceed in parallel as long as they do not conflict.

\paragraph{HashSet and HashMap}
The GlobalHashSet and GlobalHashMap have the same synchronization strategy (serialization happens at each hashed location) which allows them to scale fairly well even without combining because serialization only happens on conflicts. However, after 32 nodes, scaling drops off significantly due to increased number of concurrent accessors and more destinations.
The combining version is able to perform repeated inserts and repeated lookups with a single remote operation, enabling it to continue scaling further.

\subsection{Application Kernel Performance}
The original goal was to have scalable global data structures in the Grappa library that could be used in applications. Naive implementations of the data structures are insufficient for scaling, so HPC applications often implement their own highly customized versions. However, this forces each application to carefully reason about the ways in which the structures will be accessed. Additionally, many optimizations rely on relaxing global consistency in a way that does not affect the program. 
The Grappa data structures are synchronized to provide the most general use and match the expectations of programmers and algorithms.

\paragraph{Breadth-First Search}
\begin{figure}[t]
  \centering
  \includegraphics[width=0.5\textwidth]{data/plots/bfs_perf.pdf}
  \caption{\emph{BFS} on a Graph500-spec graph of scale 26 (64 million vertices, 1 billion edges), with the direction-optimizing BFS algorithm. Performance is measured in millions of Traversed Edges Per Second (MTEPS).}
  \label{fig:bfs_perf}
\end{figure}
The first application kernel is the Graph500 Breadth-First-Search (BFS) benchmark. This benchmark does a search starting from a random vertex in a synthetic graph and builds a search tree of parent vertices for each vertex traversed during the search. While this is a relatively simple problem, it exercises the random-access throughput of a system as well as being a primitive in many other graph algorithms. The BFS algorithm contains a global queue which represents the frontier of vertices to be visited in each level.
Our implementation employs the direction-optimizing algorithm by Beamer et al.\cite{Beamer:Graph500} which performs particularly well for the scale-free RMAT graphs generated by the benchmark.
The frontier queue in BFS is amenable to further optimization to take advantage of the fact that the algorithm does pushes and pops in separate phases, allowing consistency to be relaxed.
We compare our implementation of BFS using the flat-combined global queue described above with a highly tuned Grappa implementation that uses a custom asynchronous queue.

Figure~\ref{fig:bfs_perf} shows the results of scaling the BFS kernel up to 64 nodes. The simple queue implementation without flat combining is completely unscalable. However, the flat combining queue tracks the highly tuned asynchronous version. This illustrates that providing a safe, synchronized data structure for initially developing algorithms for PGAS is possible, while further optimizations can be applied incrementally.

\paragraph{Connected Components}
\begin{figure}[t]
  \centering
  \includegraphics[width=0.5\textwidth]{data/plots/cc_perf.pdf}
  \caption{\emph{Connected Components} on the same scale 26 Graph500 graph. Performance is measured in MTEPS.}
  \label{fig:cc_perf}
\end{figure}
Connected Components (CC) is another simple graph analysis kernel that illustrates another use of global data structures in irregular applications. We implement the three-phase CC algorithm~\cite{mtgl} which was designed for the massively-parallel MTA-2 machine. In first phase, the algorithm begins many traversals in parallel from random starting vertices, labeling vertices with the root vertex. Whenever two traversals encounter each other, their searches are pruned and an edge between the two roots is inserted in a set. After all edges have been traversed in this way, the set of edges forms a new, typically much smaller, graph. The second phase performs the classical Shiloach-Vishkin parallel algorithm\cite{shiloach1982n} on this reduced graph, and the final phase propagates the component labels out to the full graph.
The creation of the reduced-graph edge set dominates the runtime of this algorithm, so improving the implementation of the set operations has a significant impact on performance. As in the case of BFS, further optimizations involving relaxation of consistency can be applied, in this case, to the global set. Therefore, we compare our straightforward implementation using the generic GlobalHashSet with and without flat combining against a tuned asynchronous implementation.

The results in Figure~\ref{fig::cc_perf} show that none of these three implementations scale very well out to 64 nodes. However, performing combining does improves the performance of the algorithm overall and improves scaling. The tuned version outperforms the synchronous version because it is able to build up most of the set locally on each core before merging them at the end of the first phase. An implementation of the GlobalHashSet that did not provide synchronized semantics could potentially relax consistency in a more general way, but this is left for future work.

\todo{End eval insert}

\section{Related work}


hardware:

XMT/MTA

cyclops

niagra

software:

qthreads

cilk? tbb?

why not wait for more cores?

hyperthreading?



\section{Conclusion}

Large irregular problems found in areas such as social networking and
bioinformatics tend to have poor performance on commodity clusters due
to frequent low-locality memory accesses. We present a programming
model and runtime system that uses lightweight context switching to
tolerate long-latency accesses and increase performance on these
applications. Our preliminary evaluation shows that the overhead of
context switching is not prohibitive for thread counts needed for a
large-scale implementation, but memory concurrency is limited by
hardware structures in commodity processors. We describe how we plan
to address the challenges of large-scale memory concurrency and
lightweight synchronization.


\bibliographystyle{acm}
\bibliography{softxmt-hotpar2011}

\end{document}
