%% move stuff to defs file
\newcommand{\todo}[1]{{\bfseries [[#1]]}}
%% To disable, just uncomment this line
 %%\renewcommand{\todo}[1]{\relax}

\documentclass{acm_proc_article-sp}
\usepackage{url}
\usepackage{fancyvrb}
\usepackage{alltt}
\usepackage{comment}
\begin{document}

\title{Can We Crunch the Social Graph on the Cheap?}

\numberofauthors{7}
\author{
Jacob Nelson, Brandon Myers, Andrew Hunter, Dan Grossman, Luis Ceze\\
University of Washington\\
\email{\{nelson, bdmyers, ahh, djg, luisceze\}@cs.washington.edu}\\ \\
Simon Kahan \\ 
Pacific Northwest National Lab\\
\email{\{skahan\}@cs.washington.edu}
}

\maketitle
\begin{abstract}

 Techniques from graph analytics have found important applications in
  areas such as social networks and bioinformatics. These irregular
  problems present a challenge for parallel machines: obtaining
  performance is not straightforward. Multithreading works well for
  such problems, but large-scale implementations of the technique like
  the Cray XMT are expensive, non-commodity machines.

  Our goal is to build a system with XMT-like properties, implemented
  with commodity processors. This paper presents a runtime for latency
  tolerance, along with our plans for scaling up to multiple
  nodes. Our analysis shows the feasibility of our approach.
\end{abstract}

\section{Introduction}


Techniques from graph analytics have found important applications in
areas such as social networks and bioinformatics. \todo{more!}


% In January 2011, Facebook had 500 million active users with an average
% of 130 friends each \cite{Facebook:2011p91}.
% Facebook's friend suggestion
% algorithm in 2010 ran on a 40-node cluster with 72GB memory per
% node. Even with that capacity, new suggestions could only be computed
% every two days \cite{Backstrom:2010p90}.

The most interesting computational challenge comes from large,
low-diameter, power law graphs: this combination makes extracting
performance difficult. Their size keeps them from fitting in a single
commodity machine's memory. Their low diameter means that they are
difficult to lay out with locality. Their power law distribution means
that a few vertices generate much more work than the rest, leading to
computational imbalance and even less locality. Together, these
properties lead to a difficult conclusion: each edge traversal is
likely to be a very expensive cache miss.

Multithreading is a technique that has been used successfully to
implement efficient computations for these graphs. The Cray XMT is an
example of such an approach: it solves the memory latency problem
through concurrency rather than caching. Each XMT processor supports
128 hardware contexts and 1024 outstanding memory operations, and is
able to switch contexts every cycle. This ability comes at a cost: the
XMT is an expensive, non-commodity machine.

We want the best of both worlds. Our goal is to build a system that
has good performance on these low-locality graph codes but is
implemented with commodity processors. The core of our approach is a
lightweight coroutine library designed to overlap long-latency memory
prefetches with other contexts' computation.

In this paper, we describe our single-node runtime for latency
tolerance, along with our plans for scaling to multiple nodes.  We
evaluate the performance of our runtime using a set of list traversal
benchmarks designed to model certain worst-case behaviors of these
large graph problems. 

The rest of the paper is organized as follows. We present related work
in section~\ref{sec:related}. We describe the programming model we are
implementing in section~\ref{sec:model}, and describe our approach for
implementing this model in section~\ref{sec:approach}. We evaluate our
latency-tolerant runtime in section~\ref{sec:evaluation} and conclude
in section~\ref{sec:conclusion}.

\section{Related work}
\label{sec:related}
The concept of using independent parallel contexts to tolerate latency
is not novel.  The inspiration for our project is of course the Cray
XMT \cite{feo-xmt}, which implemented a hardware version of our
programming model (in what would today be considered exceptionally
slow silicon.)  The XMT can be seen as simultaneous multithreading
\cite{tullsen-smt} writ large; the technique and motivation is
essentially the same, but the XMT allows hundreds of contexts (modern
Intel SMT implementations are generally limited to two.) The Alewife
system \cite{agarwal-alewife} also switches contexts (in hardware) to
hide latency on remote-node accesses.  Cyclops \cite{almasi-cyclops}
allows hundreds of independently executing contexts to share FPUs,
caches, and paths to memory, but does not context switch or share
basic execution hardware.  

Modern programmable GPUs, such as nVidia's CUDA programming model, can also be interpreted in using an XMT-like
model under the hood;
% Do I need to include a citation here?
while they generate large numbers of available execution contexts via
a (semi-)explicit SIMD model, those SIMD instructions keep functional
units satisfied by overlapping execution with the loads required for
other contexts.

Similar techniques have also been used in software.  Mowry proposed a system \cite{mowry-scm} of \emph{software-controlled
  multithreading}.  As opposed to our system of explicitly marked
remote references, SCM assumes a high hit rate and traps to an
interrupt routine which performed a context switch upon a cache miss.  The 2000-era processors had
short enough latencies that using more than two contexts generally
produced slowdown.

Partitioned global address space models, such as that used in Unified Parallel C,
% definitely need a citation here
 share our approach of exposing all data (even remote data)
 transparently to any threads and have largely solved the problems of
 how to efficiently translate those remote reads into message
 passing.  However, their programming models emphasize locality more
 than our work does; while a program can examine any vertex in a UPC
 program at any time, it probably shouldn't (whereas our kernels rely on our
 latency hiding technique to make common remote references fast.

The value of the information in extremely large and sparse graphs is
rapidly growing.  The GraphCT toolkit \cite{ediger-graphct} used an XMT to perform
sophisticated analysis on data from Twitter, identifying important
actors and their networks of influence, and it can scale to the full
size of graphs such as Facebook's friend network.
\section{Programming model}
\label{sec:model}

Our shared-memory programming model partitions the overall address
space into a global shared address space and per node private address
spaces.  Locality may be exploited directly by the programmer in the
node private address spaces, just as in a conventional cache-coherent
multiprocessor node.  In contrast, locality cannot be exploited in 
the global space.

Whereas in the PGAS model coherency of global copies is implictly
maintained across the system, we forego caching of globally shared
data.  A node can exploit temporal locality of data stored in
the global space only by first copying it into the node's local space.
In this way, a node can access data at random across the system while
incurring no hardware coherence overhead: consistency is enforced by
our mechanisms for explicit global synchronization.

The model promises efficiency subject to the presence of sufficient
concurrency.  Exactly how that concurrency is expressed by the
programmer is language dependent.  Our programming model serves to
augment threaded programming languages so that they become latency
tolerant on large address spaces.  Software multithreading is used to
provide latency tolerance: a computation that is about to execute a
long latency operation such as a remote memory reference initiates the
operation and then yields to the scheduler, which quickly resumes
execution of another computation.  Concurrency is also used -- via
aggregation of requests to a common node destination -- to make more
efficient use of network bandwidth and in some cases reduce bandwidth
demands.

Synchronization on locations in the node private address space is the
same as on conventional systems.  Synchronization on global addresses
works differently: we provide atomic operations such as {\tt
  int\_fetch\_add} as well as operations using full-bits, as on the
Cray XMT.  As is true for other long latency operations,
synchronization latency is tolerated by yielding to the scheduler.
Even non-blocking global atomics yield: atomic operations executed as
x86 instructions sabbotage concurrency by forcing all outstanding
memory references to complete before new ones can be issued.  By
yielding, the core can switch to another computation while the
synchronization is accomplished outside of the pipeline.  In addition,
the implementation can exploit aggregation when concurrency is
abundant to increase the throughput of global synchronization.

Thus, the programming model provides the programmer with options: express
locality within the private space, and it will be exploited to reduce
latency and bandwidth; express concurrency, and it will be used toward
tolerating latency and reducing the cost of bandwidth.

\begin {comment}
\todo{Change to discuss API:

context switches:
  yield()
  discuss coroutines

mem:
  alloc()
  read(p) 
  write(p, d)
  prefetch(p)

sync:
 {read, write} {EE, EF, FE, FF}
}

While the ideal programming model is one that facilitates easy
expression in the problem domain as well as efficient execution on the
available hardware, our programming model today is motivated by only
efficient execution: it exists as a rudimentary library for concept
demonstration, not as a language suitable for development.  The
programming model we use today is therefore only partially formed. It
exists as an augmentation of existing threaded programming models such
as pthreads or openmp that are readily available on most computer
systems.

Though not strictly necessary, for ease of exposition we imagine a
one-to-one binding between threads and cores.  Our model manifests
within any thread independently of the others as a ``fray'' \todo{for lack
of better term }.  Within the thread, a fray instance is the
instantiation of a set of coroutines and a scheduler.  Instantiation
is at user-level and the operating system sees the entire fray as as a
single user-level thread.

\todo{ describe calls that perform thread \& scheduler instantiation; and schedule activation }

Each coroutine executes non-preemptively \todo{seems like non-premptively isn't ``how" it executes, rather how it is scheduled, maybe say ``cooperatively"?}, utilizing the core to which
its ``parent'' thread has been assigned by the operating system until it
reaches a yield point.  Typically, the coroutine yields after
initiating a prefetch operation on data that the programmer or
compiler anticipates would probably not complete before the data is
referenced by a subsequent load operation.  In this way, the core
continues to execute instructions for other coroutines when otherwise
it would likely stall.  The programming model thus tolerates latency.

\todo{ describe calls to prefetch-and-yield }

When a thread frays, for the sake of expediency, a fixed portion of
its stack is divided amongst the coroutines.  Given a finite stack,
each coroutine is constrained to execute a call tree of depth known to
be finite at compilation time: recursion is not yet permitted.

\todo{ discussion of coroutines \& synchronization } Were a coroutine to
block on a synchronization variable shared with other threads, the
entire fray would suspend execution.  This can lead to deadlock when,
for example, one coroutine waits to consume from another thread that
is waiting to consume what only another coroutine in this first fray
can produce.  Instead, coroutines must yield on failed synchronization
events, spinning rather than blocking, where were they bona fide
threads, blocking might be more efficient.

\todo{ say something about ``synchronization'' within a fray }

Coroutines completing their work yield without adding themselves to
the scheduling queue.  The last coroutine to exit in this way returns
as the main thread, as in the common  fork-join model of parallelism.

\end{comment}

\section{Approach}
\label{sec:approach}

Our goal is to enable high throughput on graphs having two challenging
properties. First, the graphs' combination of sparseness and low
degree means there is little locality to be exploited. Second, the
graphs are larger than a single node's memory and do not partition
well because of low diameter \todo{enough explanation now?}. We must spread the graph across multiple nodes
and send memory operations over a network. If memory requests are
uniformly distributed over the working set, a given node will make as
many accesses to remote nodes' memory as there are remote requests to the local node's memory coming in.

We would like to run as fast as current hardware will let us. Pin
bandwidth is the tightest resource: Intel's QuickPath Interconnect
(QPI) allows approximately 100 million requests per second
\cite{quickpath:website} \todo{specifically 25.6GB/s two way, which we expect 6.4GB/s for one way for outgoing request}, and recent interconnect products allow similar request
rates, with a round-trip latency of approximately 3$\mu$s
\todo{cite}. Little's law tells us then that we must support on the
order of 300 concurrent memory operations--and thus, up to this many coroutines--to
cover this latency.

These requirements drive the design of our system. There are three key
components. First, the system needs lightweight context switches (coroutines) to be able to support many contexts--thereby increasing memory concurrency available--with minimal impact on performance. Second, a memory management subsystem must support many concurrent accesses to a global address space. Third, the system must provide synchronization at a fine-granularity \todo{this last one sounds incomplete because no reason given}.

\subsection{Lightweight context switch}
\todo{can we include in the title something that means lightweight and good scheduling?}

There are three main concerns in designing the coroutine
mechanism. First, contexts must be small, so many coroutines can be
active without monopolizing the cache. Second, context switches must
be fast, so that overhead does not swamp actual work. Third, when a
coroutine yields, we must choose a suitable coroutine to run next.

We minimize context size by treating context switches as function
calls, as in \cite{charm}. This allows us to save and restore the
minimum number of registers allowed by the ABI, and depend on the
compiler to save and restore any other registers in use. We minimize
context switch time both by keeping contexts small and by doing all
switching and scheduling in user space, as in \todo{cite other green
  threads packages}.

The choice of which coroutine to run next presents us with a
tradeoff. The simplest scheduler chooses coroutines to execute in
round-robin fashion. But this may lead to performance problems: a
coroutine might be scheduled before the event which caused it to yield (e.g., a long latency memory access)
has completed, stalling the pipeline and keeping other coroutines from
progressing. A coroutine might also be scheduled long after the event
which caused it to yield has completed; while this wouldn't block
other coroutines from proceeding, it would add unnecessary latency to
the yielding coroutine. A more intelligent scheduler might allow
coroutines to execute only when they were ready, but this could
increase the overhead of context switching. Our initial implementation
takes the simple approach, but more investigation is warranted.

\subsection{Accessing Global Memory}

% Our goal is to support many concurrent accesses to a global address
% space spread across nodes in a cluster. We can draw on hardware
% support developed for the Partitioned Global Address Space family of
% languages to accomplish this.

% We draw on libraries and network support 
% goal. Languages in the Partitioned Global Address Space family such as
% X10 and Chapel have similar goals, 


% It is also useful to access this low-locality data without disturbing
% the thread contexts and local storage in the cache. Current x86
% processors support non-temporal loads and stores. These memory
% operations are quite relevant for our problem: they are designed to
% move data without polluting the cache. Using these instructions works
% better than mapping the low-locality data as uncached, since that
% restricts the processors' ability to reorder operations.



Following the lead of the PGAS family of languages, we will use
the RDMA functionality provided by modern Infiniband networks to
access our global memory. \todo{rethink?}

These networks allow us to read and write memory directly from remote
nodes, but they do so through a low-level interface designed to be
hidden behind a library. Loads and stores to global memory must be
replaced with calls to a library. This library causes the network
interface to fetch the remote data and place it in a temporary
buffer. When the requesting coroutine is scheduled again, data is
returned from the buffer and execution proceeds.

\subsubsection{Software management}
In order to minimize memory access overhead, the memory operation
library calls do not interact with the network interface
directly. Instead, they queue their operations to another thread, the
{\em memory manager thread}, running on another core in the CPU. This thread
translates the global address into a network destination and issues
the network operation, checks for completion, and manages the returned
data buffers.

But we have to be careful---directing memory requests from multiple
compute threads to a single memory manager thread may reduce memory
concurrency. And we have increased the overhead of a single memory
operation: what in the single node case was a simple read, is now
multiple queuing operations between threads and multiple reads and
writes over QPI to the network interface. Adding more memory manager
threads may enable more memory concurrency, but the overhead is still high \todo{why? because lots of code to run? so accel by impl in hardware. Maybe this is answered by the "what in...is now" already}

\subsubsection{Hardware management}
\todo{might want to call it accelerator throughout after saying FPGA once}
Considering this overhead, we propose hardware acceleration with a {\em memory manager
  FPGA}, a coprocessor FPGA inserted into a Xeon socket so it can
participate in the node's coherence domain. This coprocessor does no
computation---it would only manage the global memory requests of the
main CPU, performing the function of the memory manager thread
described earlier.

By participating in the coherence domain, the memory manager FPGA can
act as a proxy for the rest of global memory. It advertises the
address space of the rest of the system, and translates local loads
and stores to addresses on remote nodes into the network operations
required.

Unfortunately, current processors do not provide sufficient hardware
resources for us to reach our concurrency target using this
approach. Each core in Intel's X5650 Xeon supports only 10 outstanding
misses from its private L2; the entire chip is limited to 32 L2 misses
to local memory and 12 L2 misses for memory reached over the QPI
bus. A simple translation of a local load to a network access would
tie up one of these slots for the entire multi-microsecond latency of
the request. To allow the amount of concurrency we need, we must bypass these structures; the CPU needs to direct the memory manager FPGA
to do work on its behalf without tying up its own resources.

To accomplish this, we encode commands in the high-order bits of
remote memory addresses. The memory manager FPGA advertises each
remote address twice \todo{this sounded like an active thing the FPGA is doing each time you make remote request, first time I read it. Consider just sticking to cmd as upper order address bits than saying advertise each address twice}. The first address is used for blocking reads and
writes---any access incurs the full latency of the remote
access. \todo{verify writes block?} The second address is used for
prefetches and asynchronous writes. These addresses differ by one
high-order bit, easily set with a mask in a macro. A read from the
prefetch address starts the network transaction, but immediately
returns a dummy value. When the value arrives, it is stored in a
temporary buffer; when the requesting coroutine is scheduled again, it
executes the blocking read, and the memory manager immediately returns
the value from the buffer \todo{is this necessarily restricted to the memory manager returning value from buffer rather than buffer in local memory?  maybe it is, since if buffer was in local mem, would have to also write a full bit or make two reads: valid bit and data}. \todo{this isn't fully baked---do we want
  to include this sort of detail?}

\subsection{Synchronization}

In systems with hundreds of threads per node, lightweight
synchronization is important. While we have not implemented any
synchronization primitives yet, full-empty bits are a natural choice
given our programming model \todo{say why natural choice unless programming model section makes this point already}. Two questions then arise: at what
granularity should full-empty bits work, and what component should do
the synchronization?

We expect most data accesses to be to 64-bit words. It is possible to
support full-empty bit synchronization on arbitrary words by
allocating additional storage for the full-empty bits, and translating
each atomic full-empty operation into a sequence of memory operations
that modify the data and full-empty bit atomically. 

One potential optimization is to limit full-empty synchronization to
pointers to aligned 64-bit words. This leaves the bottom three bits
free to store the full-empty bits. These bits would be masked out when
the pointer was returned to the user.

These techniques can be implemented in software; \cite{qthreads} is
one example. Alternatively, atomic operations could be delegated to a
memory manager FPGA as described in the previous section. The address
command encoding would be extended to support the various full-empty
operations.

It might seem natural to use the FPGA as a memory controller as
well, storing all full-empty-bit-capable memory in DRAMs attached to
the FPGA. While this is conceptually simple, it does make the
programming model and the design more complicated. Another possibility
would be to store only the full-empty bits in memories attached to the
FPGA, and leave data in memory attached to the CPU. We believe that
starting simple is best: both data and full-empty bits should be
stored in the CPU's memory. \todo{discuss QPI bandwidth differences}

% 2TB global memory => 68GB of distributed FE bits.  This seems reasonable, given that a 2TB system might have 64 nodes, so ~1GB per memory manager if one manager per node.

\section{Evaluation}
\label{sec:evaluation}

To evaluate the feasibility of our proposal, we ran pointer chasing
benchmarks on a single-node implementation of our runtime. These
pointer chasing experiments are intended to model a particular
``wost-case'' behavior of irregular applications, where each memory
reference causes a cache miss.

We ran these experiments on a Dell PowerEdge R410 with two Xeon X5650
processors and 24GB of RAM, with hyperthreading disabled. These
processors use a NUMA memory architecture, where each processor has
its own integrated memory controller and DIMMs; references to other
processors are carried over Intel's cache-coherent QuickPath
Interconnect (QPI).

Our evaulation consists of three parts. First, we describe the
performance of the test machine's memory system to use as a
baseline. Then, we demonstrate that the same performance is achievable
using our runtime system: we measure pointer chasing performance and
the effects of cache pressure due to the number of contexts
required. Finally, we simulate the effects of fetching memory over a
network by adding delays to our runtime.

\subsection{Memory system performance}

We first characterize the performance of our test machine's memory
system. We measured three parameters: the maximum random reference
rate that can be issued by the cores in one processor, the maximum random
reference rate that can be serviced by one processor's memory controller,
and the maximum random reference rate that can be carried over the QPI
link and serviced by another processor's memory controller.

To find the maximum random reference rate that one processor's cores
can issue, we ran pointer chasing code following the model of
Figure~\ref{fig:pointernocoro}. Each core issues $n$ list traversals
in a loop; we call $n * \textit{number of cores}$ the number of
{\em concurrent offered references}, since the memory system may not
be able to satisfy them all in parallel. Since our goal is to find a
baseline for evaluating our coroutine library, we depend on the
processor's ILP for memory concurrency, rather than our coroutine
library.

\begin{figure}
\begin{alltt}
  while (count-- > 0) \{
    list1 = list1->next;
    list2 = list2->next;
    \ldots
    list\(n\) = list\(n\)->next;
  \}
\end{alltt}
\caption{Pseudocode for pointer chasing without coroutines.}
\label{fig:pointernocoro}
\end{figure}


The lists were sized to exceed the last level cache and were laid out
randomly with pointers spaced at a cache line granularity, maximizing
the probability of each reference being a miss. We allocated the lists
on 1GB huge pages to minimize TLB refill overhead. The lists were
allocated in the memory attached to the processor containing the cores
doing the traversal. We ran 10 traversals, measuring average rate in each, and recorded the maximum rate.


\begin{figure}[h]
	\begin{center}
		\includegraphics[width=0.5\textwidth]{figures/multi-listwalk-totalconc-edited.pdf}
	\end{center}
	\caption{Throughput of listwalk versus total number of
          concurrent references offered by the cores. Data points for
          running 1 core are shown as circles, and data points for
          running 6 cores are shown as diamonds. 
%Notice that the throughput for a single core levels out at 10
%concurrent references. For 4 to 6 cores, the throughput levels off at
%around 36 references, which seems to be the most memory concurrency a
%processor can handle.
        }
	\label{fig:listwalk-totalconc}
\end{figure}

Figure~\ref{fig:listwalk-totalconc} shows the result. Each point represents
the maximum rate pointers are traversed for a given number of
concurrent offered references. We see a maximum rate of 277 million
references per second, which agrees with the measurements found by \cite{Mandal:2010} with a similar machine. This rate is achieved when the number of offered
references is 42 \todo{yes, max is achieved here but essentially levels out earlier like 36; do we want a percentage metric for defining a plateau}. Note that a single core cannot support this level of
memory concurrency; the maximum reference rate for a single core is
107M. We believe this limit is due to the core's {\em line fill
  buffer} \todo{cite nehalem perf docs}, which limits the core to 10
concurrent L2 misses.

The memory controller in the processor has more bandwidth than its
cores can saturate. To measure the memory controller's maximum random
reference rate, we extended the previous experiment so that cores in
both processors were traversing lists allocated in the first processor's
memory. With this configuration, we observed a maximum rate of 360
million references per second. We believe this difference is due to
the configuration of the processor's {\em Global Queue} \todo{cite nehalem
  docs}, which limits the number of concurrently-executing read misses
from a processor's cores to its local last-level cache and memory to 32.

% take this figure out if we need the space
% the main point is room for remote references to local memory
\begin{figure}[h]
  \begin{center}
    \includegraphics[width=0.5\textwidth]{figures/qpi_bw-edited.pdf}
  \end{center}
  \caption{Pointer chasing in a remote processor's memory.}
  \label{fig:listwalk-qpi}
\end{figure}

Finally, we measured the rate at which the cores can make requests
over the QPI link. For this experiment, we allocated the lists in the
second processor, and ran the traversals on the first
processor. Figure~\ref{fig:listwalk-qpi} shows the result. We found that use
of the QPI link limits us to 175 million references per second.

\subsection{Runtime performance}

To characterize the performance of our runtime, we investigated three
effects: the maximum reference rate using coroutines
to obtain memory concurrency, the effects of cache pressure from the
coroutines' context storage, and the maximum reference rate obtainable
using a simulated network delay.

\begin{figure}
\begin{alltt}
  while (count-- > 0) \{
     prefetch(&(list1->next));
     prefetch(&(list2->next));
     \ldots
     prefetch(&(list\(n\)->next));
     switch();
     list1 = read(&(list1->next));
     list2 = read(&(list2->next));
     \ldots
     list\(n\) = read(&(list\(n\)->next));
 \}
\end{alltt}
\caption{Pseudocode for pointer chasing using coroutines.}
\label{fig:pointercoro}
\end{figure}

To find the maximum reference rate obtainable using our coroutine
library, we modified our pointer chasing benchmark as shown in
figure~\ref{fig:pointercoro}. Now there are two sources of offered
concurrent references: the ILP exploited by the processor, and the
memory concurrency enabled by prefetching and switching to a new
coroutine. As with our first experiment, we allocated the lists in the
same processor as the cores doing the traversal. 

\begin{figure}[h]
  \begin{center}
    \includegraphics[width=0.5\textwidth]{figures/multi-green-edited.pdf}
  \end{center}
  \caption{Pointer chasing with coroutines, with one reference per
    coroutine.}
  \label{fig:multi-green}
\end{figure}

Figure~\ref{fig:multi-green} shows the result. For this
experiment, we limited memory concurrency to one concurrent miss per
coroutine, so the only source of memory concurrency is the use of
coroutines. We are able to obtain a rate of 275 million references per
second with 48 concurrent misses, or 8 coroutines per core. These
results are very close to the ILP-only experiment.

We observe a gradual decrease in reference rate once the number of
concurrent references per core exceeds 10; we believe this is due to
later prefetches squashing earlier ones in the line fill buffer. We
also observe some repeatable outliers below the maximum rate even with enough
memory concurrency; this is harder to explain, but we suspect it is
due to resource contention in the cores' pipelines once the memory
system is full of requests.

\begin{figure}
\begin{alltt}
  while (count-- > 0) \{
    prefetch(&(list1->next));
    switch();
    list1 = read(&(list1->next));
    for( i in 1 to num_local_updates ) \{
      local->data++;
      local = local->next;
   \}
  \}
\end{alltt}
\caption{Pseudocode for pointer chasing with local updates.}
\label{pointerupdate}
\end{figure}

The stacks for the coroutines are stored in the data cache, where they
will compete for space with an application's data. To characterize the
effects of this cache pressure, we modified our list chasing benchmark
to include random updates to a per-coroutine local data structure that
is small enough to fit in cache. Figure~\ref{pointerupdate} shows the
general idea.


\begin{figure}[h]
  \begin{center}
    \includegraphics[width=0.5\textwidth]{figures/cache-pressure-edited.pdf}
  \end{center}
  \caption{Cache pressure with coroutines.}
  \label{fig:cache-pressure}
\end{figure}

We ran this code on six cores with all data allocated in the same
processor's memory. We varied the size of the local working set from 2KB
to 4MB, and ran both 0 and 32 local updates for each pointer
traversal. We measured the overall reference rate, including both the local updates and remote references.

Figure~\ref{fig:cache-pressure} shows the result. On the left of the
figure, we see the list-traversal-only case, where we allocate
per-coroutine space but make no references to it. As we would expect,
the reference rates grow with the number of coroutines just as in
Figure~\ref{fig:multi-green}. On the right of the figure, we see the
case with 32 updates for each list traversal. While performance
decreases for large working set sizes and large number of coroutines,
we see speedup as number coroutines increases for up to 128KB working set size with 8 coroutines per
core.

Finally, to simulate the performance of pointer chasing in a
multi-node system, we modified the scheduler of our coroutine library
to include a delay before a coroutine can be rescheduled, imitating
the network transit delay even though we are referencing local
memory. Recently published measurements of Infiniband latencies
\todo{cite} suggest we might expect 1.1 $\mu$s source-to-destination
latencies between two computers connected back-to-back; we assume a
switch transit delay of 400 ns in each direction and thus estimate a 3
$\mu$s round-trip delay. In this experiment, once a coroutine has
yielded, we use the core's timestamp counter to ensure it does not get
rescheduled for 3 $\mu$s.

\begin{figure}[h]
  \begin{center}
    \includegraphics[width=0.5\textwidth]{figures/multi-green-delay7800-edited.pdf}
  \end{center}
  \caption{Pointer chasing with simulated network delay.}
  \label{fig:network-delay}
\end{figure}

We ran these experiments with two sources of memory concurrency: each
core ran up to 167 coroutines, and each coroutine made between 1 and 4
concurrent references. The results are shown in
Figure~\ref{fig:network-delay}. We are still able to reach a
near-maximum rate of 273 million references per second, but at least 3
concurrent references per coroutine are required. Running with a single
reference per coroutine achieves 236 million references per
second. Any network interface would be connected through a QPI link,
which we measured to have a bandwidth limit of 175 million references
per second, so our runtime appears to have the capacity to chase pointers
in a remote memory. 

\todo{do we want to include that the memory system imposes max limit so really more throughput is still achievable if there are enough ILP memory concurrency?}
\todo{may want to rerun with memory on remote core to incorporate QPI limit?}


%\todo{Begin eval insert}
%\section{Evaluation}
To evaluate the impact flat combining has on the global data structures, we ran a series of experiments to test the raw performance of the data structures themselves under different workloads, and further measured their impact on performance of two simple graph benchmarks.

Experiments were run on a cluster of AMD Interlagos processors. Nodes have 32 2.1-GHz cores in two sockets, 64GB of memory, and 40Gb Mellanox ConnectX-2 InfiniBand network cards, connected via a QLogic InfiniBand switch.

\subsection{Data Structure Throughput}
First we measured the raw performance of the global data structures on extremely simple throughput workloads.

% \begin{figure*}[t]
%   \centering
\begin{figure}[t] %{0.5\textwidth}
  \centering
  \includegraphics[width=0.5\textwidth]{data/plots/vector_perf.pdf}
  \caption{\emph{GlobalStack and GlobalQueue.}
    Results are shown on a log scale for a throughput workload performing 256 million operations with 2048 workers per core and 16 cores per node. Flat combining improves throughput by at least an order of magnitude and allows performance to scale. Matching pushes and pops enables the stack to perform even better on a mixed workload.
  }
  \label{fig:vector}
\end{figure}
  % \hspace{0.05\textwidth}
  % % ~ %add desired spacing between images, e. g. ~, \quad, \qquad etc.
  % %(or a blank line to force the subfigure onto a new line)
  % \begin{subfigure}[b]{0.45\textwidth}
  % \centering
  % \includegraphics[width=\textwidth]{data/plots/queue_perf.pdf}
  % \caption{\emph{GlobalQueue.} Same parameters as stack performance results. The queue is unable to do matching locally, but benefits from reducing the amount of synchronization that must globally serialize. The mixed workload performs worse because the current implementation serializes combined enqueue and dequeue operations.}
  % \label{fig:queue}
  % \end{subfigure}
  % \hspace{0.05\textwidth}
  
  %  
\begin{figure}[t]
  \centering
  \includegraphics[width=0.5\textwidth]{data/plots/hash_perf.pdf}
  \caption{\emph{GlobalHashSet and GlobalHashMap.}
    Results are shown for a throughput workload inserting and looking up 256 million random keys in the range 0-1024 into a global hash with 1024 cells, with 2048 workers per core and 16 cores per node.
    Performance without combining scales out to 32 nodes because synchronization happens at each hash cell, but drops off as the number of destinations increases. Due to eliminating duplicate inserts and lookups, the combining version is able to continue to scale.}
    \TODO{if time: add 'delete' operation, too (show that it doesn't affect correctness or performance)}
  \label{fig:hash_perf}
\end{figure}
  % %
  % \hspace{0.05\textwidth}
  % %
  % \begin{subfigure}[b]{0.45\textwidth}
  % \centering
  % \includegraphics[width=\textwidth]{data/plots/hashmap_perf.pdf}
  % \caption{\emph{GlobalHashMap.} Same workload as for the Set, but random integers as the values in the map. Performance matches that of the Set.} \TODO{try increasing size of value "payload", currently is just a tiny int64.}
  % \label{fig:hashmap}
  % \end{subfigure}%
  %
%   \caption{
%     \emph{Raw performance of global data structures on simple throughput workloads.}
%     Results shown for workloads with varying mixtures of writing and reading operations. All experiments were performed with 16 cores per node, and a suitable set of Grappa runtime parameters fixed for a given plot.
%     \TODO{either all with error bars or none?}
%     \TODO{combine stack/queue \& set/map plots, no point in saying the same things twice.}
%   }\label{fig:datastructs}
% \end{figure*}

\paragraph{Queue and Stack}
The GlobalQueue and GlobalStack have very similar implementations in terms of how they are synchronized. Even with Grappa's aggregation, without combining, both the stack and queue completely fail to scale because of serialization of all workers' updates on the master core.
With combining, both scale well out to 64 nodes.
On the mixed workload, the stack is able to do matching locally, allowing it to reduce the amount of communication drastically, greatly improving its performance.

The queue also benefits from reduced synchronization and batching operations, and its all-push workload performs identically to the stack's.
However, the queue is unable to do matching locally, and in fact, the mixed workload performs worse because the current implementation serializes combined enqueue and dequeue operations. This restriction could be lifted with more careful synchronization at the master core allowing enqueues and dequeues to proceed in parallel as long as they do not conflict.

\paragraph{HashSet and HashMap}
The GlobalHashSet and GlobalHashMap have the same synchronization strategy (serialization happens at each hashed location) which allows them to scale fairly well even without combining because serialization only happens on conflicts. However, after 32 nodes, scaling drops off significantly due to increased number of concurrent accessors and more destinations.
The combining version is able to perform repeated inserts and repeated lookups with a single remote operation, enabling it to continue scaling further.

\subsection{Application Kernel Performance}
The original goal was to have scalable global data structures in the Grappa library that could be used in applications. Naive implementations of the data structures are insufficient for scaling, so HPC applications often implement their own highly customized versions. However, this forces each application to carefully reason about the ways in which the structures will be accessed. Additionally, many optimizations rely on relaxing global consistency in a way that does not affect the program. 
The Grappa data structures are synchronized to provide the most general use and match the expectations of programmers and algorithms.

\paragraph{Breadth-First Search}
\begin{figure}[t]
  \centering
  \includegraphics[width=0.5\textwidth]{data/plots/bfs_perf.pdf}
  \caption{\emph{BFS} on a Graph500-spec graph of scale 26 (64 million vertices, 1 billion edges), with the direction-optimizing BFS algorithm. Performance is measured in millions of Traversed Edges Per Second (MTEPS).}
  \label{fig:bfs_perf}
\end{figure}
The first application kernel is the Graph500 Breadth-First-Search (BFS) benchmark. This benchmark does a search starting from a random vertex in a synthetic graph and builds a search tree of parent vertices for each vertex traversed during the search. While this is a relatively simple problem, it exercises the random-access throughput of a system as well as being a primitive in many other graph algorithms. The BFS algorithm contains a global queue which represents the frontier of vertices to be visited in each level.
Our implementation employs the direction-optimizing algorithm by Beamer et al.\cite{Beamer:Graph500} which performs particularly well for the scale-free RMAT graphs generated by the benchmark.
The frontier queue in BFS is amenable to further optimization to take advantage of the fact that the algorithm does pushes and pops in separate phases, allowing consistency to be relaxed.
We compare our implementation of BFS using the flat-combined global queue described above with a highly tuned Grappa implementation that uses a custom asynchronous queue.

Figure~\ref{fig:bfs_perf} shows the results of scaling the BFS kernel up to 64 nodes. The simple queue implementation without flat combining is completely unscalable. However, the flat combining queue tracks the highly tuned asynchronous version. This illustrates that providing a safe, synchronized data structure for initially developing algorithms for PGAS is possible, while further optimizations can be applied incrementally.

\paragraph{Connected Components}
\begin{figure}[t]
  \centering
  \includegraphics[width=0.5\textwidth]{data/plots/cc_perf.pdf}
  \caption{\emph{Connected Components} on the same scale 26 Graph500 graph. Performance is measured in MTEPS.}
  \label{fig:cc_perf}
\end{figure}
Connected Components (CC) is another simple graph analysis kernel that illustrates another use of global data structures in irregular applications. We implement the three-phase CC algorithm~\cite{mtgl} which was designed for the massively-parallel MTA-2 machine. In first phase, the algorithm begins many traversals in parallel from random starting vertices, labeling vertices with the root vertex. Whenever two traversals encounter each other, their searches are pruned and an edge between the two roots is inserted in a set. After all edges have been traversed in this way, the set of edges forms a new, typically much smaller, graph. The second phase performs the classical Shiloach-Vishkin parallel algorithm\cite{shiloach1982n} on this reduced graph, and the final phase propagates the component labels out to the full graph.
The creation of the reduced-graph edge set dominates the runtime of this algorithm, so improving the implementation of the set operations has a significant impact on performance. As in the case of BFS, further optimizations involving relaxation of consistency can be applied, in this case, to the global set. Therefore, we compare our straightforward implementation using the generic GlobalHashSet with and without flat combining against a tuned asynchronous implementation.

The results in Figure~\ref{fig::cc_perf} show that none of these three implementations scale very well out to 64 nodes. However, performing combining does improves the performance of the algorithm overall and improves scaling. The tuned version outperforms the synchronous version because it is able to build up most of the set locally on each core before merging them at the end of the first phase. An implementation of the GlobalHashSet that did not provide synchronized semantics could potentially relax consistency in a more general way, but this is left for future work.

%\todo{End eval insert}


\section{Conclusion}
\label{sec:conclusion}




\bibliographystyle{acm}
\bibliography{softxmt-hotpar2011}

\end{document}
