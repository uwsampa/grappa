\newcommand \nextthing {\bigskip}

\noindent {\bf Luis Ceze, Ph.D.}\\
Department of Computer Science and Engineering\\
University of Washington\\
Seattle, WA 98195-2350\\
206-543-1896\\
luisceze@cs.washington.edu\\
http://www.cs.washington.edu/homes/luisceze


%%%%%%%%%%%%%%%%%%%%%%%%%%%%%%%%%%%%%%%%%%%%%%%%%%

\nextthing
\noindent{\bf Education:}

\noindent \begin{tabular}{llll}
University of Illinois at Urbana-Champaign & Computer Science  & Ph.D. & 2007\\
University of S\~{a}o Paulo & Electrical Engineering  & M.Eng. & 2002\\
University of S\~{a}o Paulo & Electrical Engineering  & B.S. & 2000
\end{tabular}

%%%%%%%%%%%%%%%%%%%%%%%%%%%%%%%%%%%%%%%%%%%%%%%%%%

\nextthing
\noindent{\bf Appointments:}

\noindent  \begin{tabular}{lll}
University of Washington &  Associate Professor &  Sep 2012 -- present\\
University of Washington &  Assistant Professor &  Sep 2007 -- Aug 2012\\
University of Illinois at Urbana-Champaign & Research Assistant & 2002 -- 2007 \\
IBM Research, Yorktown & Intern & Summers of 2003, 2004, 2006 \\
IBM Research, Yorktown & Co-op & 2001 -- 2002
\end{tabular}

%%%%%%%%%%%%%%%%%%%%%%%%%%%%%%%%%%%%%%%%%%%%%%%%%%

% \nextthing
% \noindent{\bf Relevant Experience:}

% Most of my research has been on improving programmability of
% multiprocessor systems. As part of my thesis, I developed efficient
% and complexity effective support for speculation using signatures
% (ISCA'06). I then built upon this work to develop support for
% high-performance sequential consistency (ISCA'07). Further exploring
% signatures and memory tagging, I worked on data-centric
% synchronization with data coloring (HPCA'07) and on exposing
% signatures and their operations to software to enable better code
% analysis and optimizations (ASPLOS'08). More recently, I have been working on
% architecture support for debugging and concurrency bug avoidance
% (ISCA'08, TopPicks'09, MICRO'10). Finally, I have been working on
% fully deterministic multiprocessor systems (ISCA'08, ASPLOS'09,
% ASPLOS'10, TopPicks'10).

\nextthing
\noindent{\bf Five products most closely related to the proposed project:}
\noindent\begin{enumerate}


\item Nick Hunt, Tom Bergan, Luis Ceze, Steve Gribble, ``DDOS: Taming Nondeterminism in Distributed Systems'', {\em International Conference on
    Architectural Support for Programming Languages and Operating
    Systems ({\bf ASPLOS})}, March 2013.


\item Brandon Lucia, Luis Ceze, Karin Strauss, Shaz Qadeer and
  Hans-J. Boehm ``Conflict Exceptions: Providing Simple Concurrent
  Language Semantics with Precise Hardware Exceptions for Data
  Races'', {\em International Symposium on Computer Architecture ({\bf
      ISCA})}, June 2010.

\item Tom Bergan, Owen Anderson, Joseph Devietti, Luis Ceze, Dan
  Grossman, ``CoreDet: A Compiler and Runtime System for Deterministic
  Multithreaded Execution'', {\em International Conference on
    Architectural Support for Programming Languages and Operating
    Systems ({\bf ASPLOS})}, March 2010.


\item Jose Renau, Karin Strauss, Luis Ceze, Wei Liu, Smruti Sarangi, James Tuck and Josep Torrellas, ``Thread-Level Speculation on a CMP Can Be Energy Efficient'', International Conference on Supercomputing (\textbf{ICS}), June 2005.       Selected to appear
    in {\em IEEE Micro Top Picks.}

\item Calin Cascaval, Jose G. Castanos, Luis Ceze, Monty Denneau, Manish Gupta, Derek Lieber, Jose E. Moreira, Karin Strauss, Henry S. Warren Jr,
``Evaluation of a Multithreaded Architecture for Cellular Computing'',
International Symposium on High-Performance Computer Architecture (\textbf{HPCA}), February 2002.

\end{enumerate}

\noindent{\bf Five other significant products:}

\noindent\begin{enumerate}

\item Hadi Esmaeilzadeh and Adrian Sampson and Luis Ceze and Doug Burger,
``Neural Acceleration for General Purpose Approximate Programs'', {\em
International Symposium on Microarchitecture ({\bf MICRO})}, December 2012.

\item Brandon Lucia, Benjamin Wood, Luis Ceze, ``Isolating and
  Understanding Concurrency Errors Using Reconstructed Execution
  Fragments'', {\em Conference on Programming Language Design and
    Implementation ({\bf PLDI})}, June 2011.


\item Adrian Sampson, Werner Dietl, Emily Fortuna, Dan Gnanapragasam, Luis
Ceze and Dan Grossman, ``EnerJ: Approximate Data Types for Safe and General
Low-Power Computation'', {\em Conference on Programming Language Design and
Implementation ({\bf PLDI})}, June 2011.

\item Tom Bergan, Nick Hunt, Luis Ceze, Steve Gribble, ``Deterministic Process
Groups in dOS'', {\em Symposium on Operating Systems Design and Implementation
({\bf OSDI})}, October 2010.


\item Joseph Devietti, Brandon Lucia, Luis Ceze, Mark Oskin, ``DMP:
Deterministic Shared Memory Multiprocessing'', {\em International Conference
on Architectural Support for Programming Languages and Operating Systems ({\bf
ASPLOS})}, March 2009. Selected to appear in {\em IEEE Micro Top Picks.}


\end{enumerate}

\nextthing
\noindent{\bf Synergistic activities:}

\noindent\begin{itemize}
\item Co-Organizer, Workshop on Deterministic Multiprocessing and Parallel Programming (funded by NSF), 2009 and 2011. 

\item PC Co-chair,  USENIX Workshop on Hot Topic in Parallelism ({\bf HotPar}), 2012.

\item PC Member, IEEE Micro Top Picks, 2010, 2013, Annual Symposium on
Principles and Practice of Parallel Programming ({\bf PPoPP}) 2012,
International Conference on Architectural Support for Programming Languages
and Operating Systems ({\bf ASPLOS}) 2011, 2010, 2009, International
Conference on Microarchitecture ({\bf MICRO}) 2011, 2010, International
Symposium on Computer Architecture ({\bf ISCA}), 2008.

\item Released publicly available software: {\bf SESC}, a free cycle-accurate multiprocessor simulator (sesc.sourceforge.net); and {\bf CoreDet}, a compiler and runtime system for deterministic execution (sampa.cs.washington.edu); {\bf Recon}, a system for finding and understanding concurrency bugs. 

\item ACM Best Teacher Award, UW CSE, University of Washington, 2010
  
\end{itemize}

\nextthing
\noindent{\bf Collaborators in the last 48 months:}

\noindent\begin{tabular}{ll}
  Karin Strauss, Shaz Qadeer, Doug Burger & Microsoft Research \\
  Hans-J Boehm  & HP Labs \\
  Dan Grossman, Susan Eggers, Mark Oskin, Steve Gribble, Henry M.  Levy  & University of Washington \\
  C\u{a}lin Ca\c{s}caval & Qualcomm Research \\
  James Tuck & NC State University \\
  Josep Torrellas & UIUC 
\end{tabular}

\nextthing
\noindent{\bf Graduate advisor:} Josep Torrellas, UIUC

\nextthing
\noindent{\bf Graduate advisees:}

\noindent\begin{tabular}{l}
  Current: Tom Bergan, Joseph Devietti, Brandon Lucia, Adrian Sampson \\ 
  \hspace{1cm} Jacob Nelson, Nick Hunt, Hadi Esmaeilzadeh, Ben Wood, Katelin Bailey\\
  Past: Angda Chen (now at Amazon), Cherie Cheung, Nick Murphy (Harvard) \\
  \hspace{1cm} Owen Anderson (Apple), Emily Fortuna (Google) \\
\end{tabular}
