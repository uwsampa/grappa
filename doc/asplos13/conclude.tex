\section{Conclusion}

Irregular computations are both important and challenging to execute quickly.
Scaling these applications easily to commodity hardware has been a historical
challenge. Grappa is a runtime framework that simplifies this task for
software developers and compiler writers. This paper describes the Grappa
framework and its three main components: a task library, a distributed shared
memory system, and a network aggregator to make commodity networks efficient
with small message sizes. Grappa key aspect is extreme latency tolerance,
which not only hides network latency but also enables the system to spend time
on sophisticated work stealing and network optimizations, trading latency for
even more throughput.

We explored the performance of Grappa on three algorithms: unbalanced tree
search, breadth first search, and betweenness centrality. Performance
comparisons to the Cray XMT for a small number of nodes (16 nodes / 96 cores)
show great promise: Grappa performs on-par with the far more expensive and
custom XMT system, sometimes achieving better performance and sometimes worse.
Our initial large scaling experiments to a higher number of nodes shows both promise and room for improvement.  Grappa trounces XMT on UTS, but while Grappa performance scales on BFS and betweenness centrality, limitations in the current aggregator design give the performance edge to XMT.  Work is ongoing on further refining this component.