\section{Conclusion}

Irregular computations are both important and challenging to execute quickly. Scaling these applications easily to commodity hardware has been a historical challenge. Grappa is a runtime framework that simplifies this task for software developers and compiler writers.  This paper describes the Grappa framework and its three main components: a task library, a distributed shared memory system, and a network aggregator to make commodity networks efficient with small message sizes.  We explored the performance of Grappa on three algorithms: unbalanced tree search, breadth first search, and betweenness centrality.  Performance comparisons to the Cray XMT for a small number of nodes (16 nodes / 96 cores) show great promise: Grappa performs on-par with the far more expensive and custom XMT system, sometimes achieving better performance and sometimes worse.  Our initial scaling experiments to a higher number of nodes suggests Grappa does scale, but there is room for further improvement.  Due to limitations in the initial aggregator design Grappa cannot utilize more than 6 cores/node, and this limits not only peak performance at a small number of nodes, but scaling performance at a larger number of nodes.  Current work is ongoing on a new aggregator design.
