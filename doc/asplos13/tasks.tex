\section{Tasks}

The basic unit of execution in Grappa is a {\em task}. Each task is
represented by a function pointer and its arguments and is enqueued,
but not instantiated, when spawned.  Later, when resources are free,
the task is allocated a stack, bound to a core, and executed.

During execution, a task yields control of its core whenever it performs
a long-latency operation, allowing the processor to remain busy while
waiting for the operation to complete.  In addition,  a programmer can
direct scheduling explicitly via the Grappa API calls shown
in Figure~\ref{fig:scheduling}.
\TODO{cite UPC split phase reads and
writes, but include that we provide a way to overlap computation}

To minimize yield overhead, all Grappa scheduling is done in
user-space and saves minimal state, leading to context switch times as
low as \checkme{40ns} even when switching amongst thousands of tasks.
This fast context switching amongst many tasks is crucial to Grappa's
effectiveness for two reasons.  First, latency can be tolerated only
when the net cost of context switching is dwarfed by the loss incurred
by the alternative: stalling.  Second, because we can switch amongst
so many tasks efficiently, we can \textbf{trade latency for throughput}:
by {\em increasing} latency in key components of the
system we are able to increase our aggregate random access bandwidth,
our synchronization bandwidth, and our ability to tolerate load
imbalance.
\TODO{check to see we actually explain these three somewhere.  Ie, that aggregating increases throughput and latency, that delegating synchronization increases the rate and the latency at which we can eg atomically increment, and that stealing work increases the latency of some tasks (eg when their data is largely at the originating node) but provides greater throughput overall.}

\begin{figure}[htbp]
  \begin{center}
    \begin{description}\small
    \item[ \texttt{ yield() } ] \hfill \\
      Yields core to scheduler, enqueuing caller to be scheduled again soon
    \item[ \texttt{ suspend() } ] \hfill \\
      Yields core to scheduler, enqueuing caller only once another task calls wake
    \item[ \texttt{ wake( task * $t$ ) } ] \hfill \\
      Enqueues some other task $t$ to be scheduled again soon
    \end{description}
    \begin{minipage}{0.95\columnwidth}
      \caption{\label{fig:scheduling} Grappa API: scheduling} %{-4ex}}
    \end{minipage}
    %\vspace{-3ex}
  \end{center}
\end{figure}


\subsection{Expressing parallelism}

A programmer's goal in coding with Grappa should be to express as much
parallelism as possible without worrying about where it will execute.
Grappa then chooses where and when to exploit this parallelism,
scheduling as much work as necessary to tolerate network latencies and
keep cores busy. 

Grappa provides four methods for expressing parallelism, shown in
Figure~\ref{fig:expressing-parallelism}. The first is explicit task
spawns. When the programmer identifies work that can be done in
parallel, the work may be wrapped up in a function and queued with its
arguments for later execution using a \texttt{spawn}. If the
programmer wishes to spawn a task on a specific core in the system,
or at the home core of a particular memory location, Grappa provides a
\texttt{spawn\_on} call for this purpose.

The next method for expressing parallelism is a parallel for loop, where
the number of iterations must be known at loop entry. The programmer
specifies a function pointer along with start and end indices and an
optional threshold to control parallel overhead. Grappa does {\em
recursive decomposition} of iterations, similar to Cilk's cilk\_for
construct~\cite {cilkforimplementation} \comment{could only find slides
page6 of
http://www.clear.rice.edu/comp422/lecture-notes/comp422-2012-Lecture5-
Cilk++.pdf    ---- Good enough.  Cite it.  -Mark}.  It generates a
logarithmically-deep tree of tasks, stopping to execute the loop body
when the number of iterations is below the required threshold. Finally,
a programmer may want to run a small piece of code on a particular core
in the system without waiting for execution resources to be available.
Grappa provides the \texttt{call\_on} call for this purpose.

\begin{figure}[htbp]
  \begin{center}
    \begin{description}\small
    \item[ \texttt{spawn( void (*fp)(args) )} ] \hfill \\
      Creates a new stealable task
    \item[ \texttt{spawn\_on( core, (*fp)(args) )} ] \hfill \\
      Creates a new private task that will run on a specific core 
    \item[ \texttt{parallel\_for( (*fp)(args), start, end )} ] \hfill \\
      Executes iterations of a loop as stealable tasks 
    \item[ \texttt{call\_on( core, (*fp)(args) )} ] \hfill \\ 
      Runs a limited function on a specific core without consuming
      Grappa execution resources 
    \end{description}
    \begin{minipage}{0.95\columnwidth}
      \caption{\label{fig:expressing-parallelism} Grappa API: expressing parallelism} % \vspace{-4ex}}
    \end{minipage}
    %\vspace{-3ex}
  \end{center}
\end{figure}

\subsection{Implementation tasks}

\paragraph{Tasks and workers} Grappa tasks are 32-byte entities: a
64-bit function pointer plus three 64-bit arguments. We use three
arguments because tasks are usually generated as part of a parallel loop
decomposition, and thus each task needs three kinds of data.
\begin{description}
\item[Function pointer] This 64-bit value indicates what routine
  should run. We configure the system nodes to disable address randomization, so that we can depend on the fact that function pointers are valid across process images.
\item[Private argument] This 64-bit value is typically used for task-specific
  data (e.g., a loop index).
\item[Shared argument] This 64-bit value is typically used for data shared
  between all tasks that are part of a loop or to specify the number
  of loop iterations.
\item[Synchronization argument] This 64-bit value is typically used to determine
  when all tasks that are part of a loop have finished. It is usually
  a global pointer to a synchronization object allocated at the core
  that spawned the task.
\end{description}
While these are the most common uses of the three task arguments, they
are treated as arbitrary 64-bit values in the runtime, and can be used
for any purpose.

Tasks are not allocated any execution resources until the scheduler
decides to run them; when this occurs, tasks are matched with {\em
  worker} threads. Each worker is simply a collection of status bits and a
stack, allocated at a particular core.

\paragraph{Context switching} Grappa context switches between tasks
non-preemptively. As with other cooperative multithreading systems, we
treat context switches as function calls, saving and restoring only the
callee-saved state as specified in the x86-64 ABI \cite{someone}. This
involves saving six general-purpose 64-bit registers and the stack
pointer, as well as the 16-bit x87 floating point control word and the
SSE context/status register. Thus, the minimum amount of state a
cooperative context switch routine must save according to the ABI is 62
bytes.

Since the compiler sees all calls to the context switch routine, we
can save even less state. Our context switch routine appears to the
compiler as inline assembly; we declare all the registers we need
to save as ``clobbered'' by the inline assembly routine, and the
compiler will issue its own save and restore code as needed. This allows the
compiler to avoid saving any registers that are not used, or are used
for temporary values that are not needed after the context switch.

\paragraph{Scheduling} Each core in the Grappa system has its own
independent scheduler. Each scheduler has three main tasks to perform. 
First, it must ensure that its communication resources are serviced
reasonably often. Second, it must ensure that if a running task was
waiting on a long-latency operation and that operation has completed,
the task will be rescheduled. Third, if there are tasks waiting to be
run and spare execution resources, the tasks must be matched with idle
workers.

Each scheduler has three queues:
\begin{description}
\item{\bf Ready worker queue} This is a FIFO queue of tasks that are
  matched with workers and are ready to execute.
\item{\bf Private task queue} This is a FIFO queue of tasks that must run on this core.
\item{\bf Public task queue} This is a LIFO queue of tasks that are
  waiting to be matched with workers. It is a local partition of a shared
  task pool.
\end{description}

Whenever a task yields or suspends, the scheduler makes a decision about
what to do next. First, it determines if the communication resources
should be serviced. This is done on a periodic basis. Second, it
determines if any workers with running tasks are ready to execute; if
so, one is scheduled. Finally, if there are no workers ready to run, but
there are tasks waiting to be matched with workers, an idle worker is
woken (or a new worker is spawned), matched with a task, and scheduled.
The scheduler ensures that communication maintenance tasks are executed periodically.

\paragraph{Work stealing} When a Grappa node runs out of work, it
becomes a ``thief'' and asks another other Grappa nodes (``victim'') for
work. If the victim has tasks to spare in its public queue, they will be
transferred to the thief's public queue.

Most prior work on work stealing assumes that a processor has one worker
thread and so it only steals when utilization would go to zero. Grappa,
on the other hand, relies on having many worker threads per core. The
core would be fully-utilized if there is always something on the ready
queue. Even if there are many active tasks, if they are all suspended
for long-latency network requests, then the core is underutilized. So,
there is a choice: should the core use some of this idle time to perform
work stealing or should it just wait as it is possible that local tasks
will create more work. We choose to have a core initiate a steal when
all of the following conditions hold: no workers are ready to run, the
unstarted task queues are empty, and there are no outstanding steal
requests. Having only one outstanding steal request throttles steals to
prevent flooding the network, but other scalable quieting mechanisms
would be possible, such as a voting
tree\cite{scalableWorkStealingOrCilk98} or lifelines \cite{lifelines}.
Termination detection is not built into the Grappa task scheduler,
rather, it is considered a programming error for the program to exit
without syncing all tasks.

\TODO{Someone check this paragraph to make sure it is true. :-)  -Mark}

When a grappa node decides to issue a steal request it chooses the
victim node at random.  It also requests a block of tasks at once.  If
this block exceeds the amount of stealable tasks on the victim node,
half of the stealable tasks are stolen instead.  While we did study the
number of tasks to steal at once, we found performance was maximized
with \checkme{128} tasks with little sensitivity around this number,
hence we omit results for this from Section~\ref{sec:results} and simply
use \checkme{128} as the parameter for all experiments.



