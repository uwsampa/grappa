\section{Tasks}

The basic unit of execution in Grappa is a {\em task}. Each task is
represented by a function pointer and its arguments. A new task is enqueued at
spawn time; when resources are free, it is allocated a stack, bound to a core,
and executed.

During execution, a task yields control of its core whenever it performs
a long-latency operation, allowing the processor to remain busy while
waiting for the operation to complete.  In addition,  a programmer can
direct scheduling explicitly via the Grappa API calls shown
in Figure~\ref{fig:scheduling}.
% \TODO{cite UPC split phase reads and
% writes, but include that we provide a way to overlap computation}

To minimize yield overhead, the Grappa scheduler operates entirely in
user-space and does little more than store register state of one task and load
that of another. Context switch times are as low as 38ns even when switching
amongst thousands of tasks. Fast context switching amongst many tasks is
crucial to Grappa for two reasons. First, only when the net cost of context
switching is dominated by the loss incurred by stalling does a latency
tolerant approach work to increase throughput on computations with intense
random access. Random access latencies to node local memory are on the order
of 100ns. Were the scheduler to add more overhead than this, it would actually
slow down rather than speed up computations that randomly access node local
data. Second, because we can switch amongst so many tasks without dramatically
increasing scheduling overhead, Grappa can tolerate system latencies many
orders of magnitude greater than that of local memory. Network ping latency
alone is approximately 2us, but we can tolerate latencies in the millisecond
scale. Furthermore, by over-provisioning task parallelism, Grappa promises to
provide robust throughput that is relatively insensitive to variations in
latencies within and across our target systems.

\begin{figure}[htbp]
  \begin{center}
    \begin{description}\small
    \item[ \texttt{ yield() } ] \hfill \\
      Yields core to scheduler, enqueuing caller to be scheduled again soon
    \item[ \texttt{ suspend() } ] \hfill \\
      Yields core to scheduler, enqueuing caller only once another task calls wake
    \item[ \texttt{ wake( task * $t$ ) } ] \hfill \\
      Enqueues some other task $t$ to be scheduled again soon
    \end{description}
    \begin{minipage}{0.95\columnwidth}
      \caption{\label{fig:scheduling} Grappa API: scheduling} %{-4ex}}
    \end{minipage}
    %\vspace{-3ex}
  \end{center}
\end{figure}


\subsection{Expressing parallelism}

Grappa programmers focus on expressing as much
parallelism as possible without concern for where it will execute.
Grappa then chooses where and when to exploit this parallelism,
scheduling as much work as is necessary on each core to keep it busy
in the presence of system latencies and task dependences.

Grappa provides four methods for expressing parallelism, shown in
Figure~\ref{fig:expressing-parallelism}:

First, when the programmer identifies work
that can be done in parallel, the work may be wrapped up in a function
and queued with its arguments for later execution using a
\texttt{spawn}.

Second, a programmer can use \texttt{spawn\_on} to spawn a task on a
specific core in the system or at the home core of a particular memory
location.

Third, the programmer can invoke a parallel for loop, provided that
the trip count is known at loop entry. The programmer
specifies a function pointer along with start and end indices and an
optional threshold to control parallel overhead. Grappa does {\em
recursive decomposition} of iterations, similar to Cilk's cilk\_for
construct~\cite {cilkforimplementation}.\comment{could only find slides
page6 of
http://www.clear.rice.edu/comp422/lecture-notes/comp422-2012-Lecture5-
Cilk++.pdf    ---- Good enough.  Cite it.  -Mark} and TBB's {\tt parallel\_for}~\cite{intel_tbb}.  It generates a
logarithmically-deep tree of tasks, stopping to execute the loop body
when the number of iterations is below the required threshold.

Fourth, a programmer may want to execute an active message; that is,
to run a small piece of code on a particular core in the system
without waiting for execution resources to be available.  For example,
custom synchronization primitives execute this way, as a function
executed on the core where the data lives.  Grappa provides the
\texttt{call\_on} call for this purpose.

\begin{figure}[htbp]
  \begin{center}
    \begin{description}\small
    \item[ \texttt{spawn( void (*fp)(args) )} ] \hfill \\
      Creates a new stealable task
    \item[ \texttt{spawn\_on( core, (*fp)(args) )} ] \hfill \\
      Creates a new private task that will run on a specific core 
    \item[ \texttt{parallel\_for( (*fp)(args), start, end )} ] \hfill \\
      Executes iterations of a loop as stealable tasks 
    \item[ \texttt{call\_on( core, (*fp)(args) )} ] \hfill \\ 
      Runs a limited function on a specific core without consuming
      Grappa execution resources 
    \end{description}
    \begin{minipage}{0.95\columnwidth}
      \caption{\label{fig:expressing-parallelism} Grappa API: expressing parallelism} % \vspace{-4ex}}
    \end{minipage}
    %\vspace{-3ex}
  \end{center}
\end{figure}

\subsection{Implementation tasks}

\paragraph{Tasks and workers} Grappa tasks are 32-byte entities: a
64-bit function pointer plus three 64-bit arguments. We use three
arguments because tasks are often generated via a parallel loop
decomposition, in which each task needs three kinds of data.
\begin{description}
\item[Function pointer] addresses the routine to run. We configure the system nodes to disable address randomization, so that function pointers are valid across process images.
\item[Private argument] often a loop index.
\item[Shared argument] typically data shared amongst a group of tasks, or the number of loop iterations to be performed by this task
\item[Synchronization argument] often used to determine
  when all tasks that are part of a loop have finished; a global pointer to a synchronization object allocated at the core that spawned a group of tasks.
\end{description}
While these are the most common uses of the three task arguments, they
are treated as arbitrary 64-bit values in the runtime, and can be used
for any purpose.

Tasks are not allocated any execution resources until the scheduler
decides to run them; when this occurs, tasks are matched with {\em
  worker} threads. Each worker is simply a collection of status bits and a
stack, allocated at a particular core.

\paragraph{Context switching} Grappa context switches between tasks
non-preemptively. As with other cooperative multithreading systems, we
treat context switches as function calls, saving and restoring only the
callee-saved state as specified in the x86-64 ABI \cite{amd64:abi:2012}. This
involves saving six general-purpose 64-bit registers and the stack
pointer, as well as the 16-bit x87 floating point control word and the
SSE context/status register. Thus, the minimum amount of state a
cooperative context switch routine must save according to the ABI is 62
bytes.

Since the compiler sees all calls to the context switch routine, we
can save even less state. Our context switch routine appears to the
compiler as inline assembly; we declare all the registers we need
to save as ``clobbered'' by the inline assembly routine, and the
compiler will issue its own save and restore code as needed. This allows the
compiler to avoid saving any registers that are not used, or are used
for temporary values that are not needed after the context switch.

\paragraph{Scheduling} Each core in the Grappa system has its own
independent scheduler. Each scheduler has three main tasks to perform:
servicing communication requests as described in
Section~\ref{sec:communication}; rescheduling tasks that have been
waiting on long-latency operations; and assigning ready tasks to
worker resources that have become idle.

Each scheduler has three queues:
\begin{description}
\item{\bf Ready worker queue} This is a FIFO queue of tasks that are
  matched with workers and are ready to execute.
\item{\bf Private task queue} This is a FIFO queue of tasks that must run on this core.
\item{\bf Public task queue} This is a LIFO queue of tasks that are
  waiting to be matched with workers. It is a local partition of a shared
  task pool.
\end{description}


Whenever a task yields or suspends, the scheduler makes a decision
about what to do next. Servicing communication requests is given priority
to ensure responsiveness, but to minimize overhead should context switches be frequent, servicing is performed only if sufficient time has elapsed.
Second, the scheduler determines if any workers with running tasks are
ready to execute; if so, one is scheduled. Finally, if there are no
workers ready to run, but there are tasks waiting to be matched with
workers, an idle worker is woken (or a new worker is spawned), matched
with a task, and scheduled.

\paragraph{Work stealing} 

When the scheduler finds no work to assign to its workers, it
commences to steal work from other cores using an asynchronous
\tt{call_on} active message.  It chooses a victim at random until it
finds one with a non-zero amount of work in its public task queue.  Up
to a tunable limit, the scheduler steals half of the tasks it finds.
We did study the impact of this limit on performance and found
\checkme{128} tasks to be optimal with little sensitivity around this
number.  Hence we omit results for this study from
Section~\ref{sec:results} and simply use \checkme{128} as the
parameter for all experiments.

\TODO{Someone check this paragraph to make sure it is true. :-)  -Mark}
\TODO{Anything after this I'm not sure adds to the paper -- if it does, please edit -Simon}
The
core would be fully-utilized if there is always something on the ready
queue. Even if there are many active tasks, if they are all suspended
for long-latency network requests, then the core is underutilized. So,
there is a choice: should the core use some of this idle time to perform
work stealing or should it just wait as it is possible that local tasks
will create more work. We choose to have a core initiate a steal when
all of the following conditions hold: no workers are ready to run, the
unstarted task queues are empty, and there are no outstanding steal
requests. Having only one outstanding steal request throttles steals to
prevent flooding the network, but other scalable quieting mechanisms
would be possible, such as a voting
tree\cite{scalableWorkStealingOrCilk98} or lifelines \cite{lifelines}.
Termination detection is not built into the Grappa task scheduler,
rather, it is considered a programming error for the program to exit
without syncing all tasks.


