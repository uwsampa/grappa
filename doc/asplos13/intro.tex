\section{Introduction} \label{sec:intro}

Irregular applications generate tasks with work, interdependences, or memory
accesses that are highly sensitive to input. Classic examples of irregular
applications include branch and bound optimization, SPICE circuit simulation,
contact algorithms in car crash analysis, and network flow. Important
contemporary examples include processing large graphs in the business,
national security, machine learning, data-driven science, and social network
computing domains. For these emerging applications, fast response -- given the
sheer amount of data -- requires multinode systems. The most broadly available
multinode systems are those built from x86 commodity computing nodes
interconnected via ethernet or infiniband. Our goal is to enable scalable
performance of irregular applications on these mass market systems.

Our goal is challenging for two key reasons: 

\noindent{\bf Irregular applications exhibit little spatial locality.}
It is not atypical for any given task's data references to be spread
randomly across the entire memory of the system. This makes current
memory hierarchy features ineffective. Caches are of little assistance
with such low data re-use and spatial locality. Commodity prefetching
hardware is effective only when addresses are known many cycles before
the data is consumed or the accesses follow a predictable pattern,
neither of which occurs in irregular applications. As a consequence,
commodity microprocessors stall often when executing irregular
applications.

\noindent{\bf Irregular applications frequently request small amounts of off-node data.} On multinode systems, the challenges presented by low
locality are analogous, and exacerbated by the increased latency of going
off-node. Irregular applications also present a challenge to mass market
network technology, which is designed to transfer large blocks of data, not
the word-sized references emitted by irregular application tasks. Injection
rate into the network is insufficient to utilize wire bandwidth when blocks
are below about two-kilobytes, so any straightforward communication strategy
severely under-utilizes the network. 

While some irregular applications can be restructured to better exploit
locality, aggregating requests to increase message size, and manage the
additional challenges of load balance and synchronization across multinode
systems, the work to do so is formidable and requires knowledge and skills
pertaining to distributed systems far beyond those of most application
programmers.

Luckily, many of the important irregular applications (e.g., graph
processing, our focus in this paper) naturally offer large amounts of
concurrency. This immediately suggests taking advantage of concurrency
to tolerate the latency of data movement. The fully custom Tera
MTA-2~\cite{tera:mta1} system is a classic example of supporting irregular
applications by using concurrency to hide latencies. It had a large
distributed shared memory with no caches.  On every clock cycle, each
processor would execute a ready instruction chosen from one of its 128
hardware thread contexts, a sufficient number to fully hide memory
access latency.  The network was designed with a single word injection
rate that matched the processor clock frequency and sufficient bandwidth
to sustain a reference from every processor on every clock cycle.
Unfortunately, while an excellent match to extremely irregular
applications, the MTA was not cost-effective on applications that could
exploit locality and had very poor single-thread performance, making
it a commercial failure. The Cray XMT approximates the Tera MTA-2,
substantially reducing its cost but not overcoming its narrow range of
applicability.

% Two opportunities help overcome these challenges without putting more burden on the programmer: (1) Many interesting irregular applications (e.g., graph processing, our focus on this paper) naturally observe large amounts of concurrency; (2) 

% In addition to large amounts of concurrency in irregular applications, 
% 
% Commodity systems offer a large number of cores with good single-thread performance. Concurrency can be used to hide network latencies involved in fetching data from remote notes, mitigating the lack of locality. 
% 
% 
% ; (3) Commodity networks improved significantly and offer high bandwidth at a relatively low cost

In the twenty years that have elapsed since the Tera MTA, commodity
microprocessors have become much faster and multicore has driven down the
absolute price of computation; commodity network price-performance has
improved as well. This shift has afforded us the opportunity to attack the
challenges posed by irregular applications by emulating in software and
inexpensive mass market hardware, the approach taken by Tera. We exploit the
increased aggregate instruction rate per socket relative to chip bandwidth,
using what would otherwise be wasted instructions to manage the multiplexing
of as many as several thousand tasks per core, thus tolerating memory latency,
reducing stalls, and making better use of available bandwidth. Ultimately, the
opportunity is to cover the spectrum of irregular to regular computation:
where tasks exhibit locality, multiplex fewer tasks and expend fewer
instructions on context switching; where locality is lacking, multiplex more
tasks at a higher rate to tolerate latency. Thus we make the best use of task
parallelism -- either to scale to more cores or to tolerate latency -- and of
caches -- either to exploit application locality or to house more task
contexts.

In this paper we introduce Grappa, a software runtime system that allows
a commodity x86 distributed-memory HPC cluster to be programmed as if it
were a single large shared-memory machine and provides scalable
performance for irregular applications. Grappa is designed to smooth
over some of the performance discontinuities in commodity hardware,
giving good performance when there is little locality to be exploited
while allowing the programmer to exploit it when it is available. 

Grappa leverages as much freely available and commodity infrastructure as
possible. We use unmodified Linux for the operating system and an
off-the-shelf user-mode infiniband device driver stack~\cite{OFED}. MPI is
used for process setup and tear down. GASnet~\cite{gasnet} is used as the
underlying mechanism for remote memory reads and writes using active message
invocations. To this commodity hardware and software mix Grappa adds three
main software components: (1) a \emph{lightweight tasking} layer that supports
a context switch in as few as 38ns and distributed global load balancing; (2)
a \emph{distributed shared memory} layer that supports customary operations
such as \emph{read and write as well as remote operations} such as
fetch-and-add~\cite{fetchandadd}; and (3) a \emph{message aggregation} layer
that combines short messages to mitigate the aforementioned problem that
commodity networks are designed to achieve peak bandwidth only on large packet
sizes, yet irregular applications tend to fetch only a handful of bytes at a
time. As we will show later, Grappa can tolerate latencies way beyond that of
the network. Therefore, Grappa can afford to \emph{trade latency for
throughput}: by {\em increasing} latency in key components of the system we
are able to increase our effective random access memory bandwidth (by delaying and aggregating messages), our synchronization bandwidth (by delegating operations
to remote nodes), and our ability to improve load imbalance (work stealing
increases latency).

% 
% Each of these components will be described in more detail in Section~\ref{sec:grappa}.
% 
% {\em Provide evidence we have succeeded -- will rewrite as necessary based on final results:}

Our evaluation of Grappa shows that it runs several graph-crunching
applications (classic examples of irregular behavior) very efficiently on a
commodity cluster. Our yardstick for comparison is the XMT hardware itself.
Using the same number of processors (in our case cores on a single multicore
processor), Grappa is in the same ballpark as the XMT: on some graphs, such as
the \checkme{WTF} Grappa is \checkme{2X} faster, while on others, such as the
\checkme{WTF2} Grappa is \checkme{2X} slower. In Section~\ref{sec:results} we
explore the factors that underpin this performance. But most importantly,
application performance \emph{scales} with Grappa. For significantly less real
world cost, users can \emph{add} more processors to a commodity cluster and
easily outperform the XMT hardware on graph processing, while still saving
significant dollars.

% 
% \TODO{Add roadmap here. Luis: do we really want a roadmap here? We have enough about what we have done I think, so we might not need to sell things we plan to do.}


