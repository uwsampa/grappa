\section{Implementation} \label{sec:implementation}

We now discuss implementation details for the major system components discussed in Section~\ref{sec:grappa}.

\subsection{Tasks}

\paragraph{Tasks and workers} Grappa tasks are 32-byte entities: a
64-bit function pointer plus three 64-bit arguments. We use three
arguments because tasks are usually generated as part of a parallel loop
decomposition, and thus each task needs three kinds of data.
\begin{description}
\item[Function pointer] This 64-bit value indicates what routine
  should run. We configure the system nodes to disable address randomization, so that we can depend on the fact that function pointers are valid across process images.
\item[Private argument] This 64-bit value is typically used for task-specific
  data (e.g., a loop index).
\item[Shared argument] This 64-bit value is typically used for data shared
  between all tasks that are part of a loop or to specify the number
  of loop iterations.
\item[Synchronization argument] This 64-bit value is typically used to determine
  when all tasks that are part of a loop have finished. It is usually
  a global pointer to a synchronization object allocated at the core
  that spawned the task.
\end{description}
While these are the most common uses of the three task arguments, they
are treated as arbitrary 64-bit values in the runtime, and can be used
for any purpose.

Tasks are not allocated any execution resources until the scheduler
decides to run them; when this occurs, tasks are matched with {\em
  worker} threads. Each worker is simply a collection of status bits and a
stack, allocated at a particular core.

\paragraph{Context switching} Grappa context switches between tasks
non-preemptively. As with other cooperative multithreading systems, we
treat context switches as function calls, saving and restoring only the
callee-saved state as specified in the x86-64 ABI \cite{someone}. This
involves saving six general-purpose 64-bit registers and the stack
pointer, as well as the 16-bit x87 floating point control word and the
SSE context/status register. Thus, the minimum amount of state a
cooperative context switch routine must save according to the ABI is 62
bytes.

Since the compiler sees all calls to the context switch routine, we
can save even less state. Our context switch routine appears to the
compiler as inline assembly; we declare all the registers we need
to save as ``clobbered'' by the inline assembly routine, and the
compiler will issue its own save and restore code as needed. This allows the
compiler to avoid saving any registers that are not used, or are used
for temporary values that are not needed after the context switch.

\paragraph{Scheduling} Each core in the Grappa system has its own
independent scheduler. Each scheduler has three main tasks to perform. 
First, it must ensure that its communication resources are serviced
reasonably often. Second, it must ensure that if a running task was
waiting on a long-latency operation and that operation has completed,
the task will be rescheduled. Third, if there are tasks waiting to be
run and spare execution resources, the tasks must be matched with idle
workers.

Each scheduler has three queues:
\begin{description}
\item{\bf Ready worker queue} This is a FIFO queue of tasks that are
  matched with workers and are ready to execute.
\item{\bf Private task queue} This is a FIFO queue of tasks that must run on this core.
\item{\bf Public task queue} This is a LIFO queue of tasks that are
  waiting to be matched with workers. It is a local partition of a shared
  task pool.
\end{description}

Whenever a task yields or suspends, the scheduler makes a decision about
what to do next. First, it determines if the communication resources
should be serviced. This is done on a periodic basis. Second, it
determines if any workers with running tasks are ready to execute; if
so, one is scheduled. Finally, if there are no workers ready to run, but
there are tasks waiting to be matched with workers, an idle worker is
woken (or a new worker is spawned), matched with a task, and scheduled.
The scheduler ensures that communication maintenance tasks are executed periodically.

\paragraph{Work stealing} When a Grappa node runs out of work, it
becomes a ``thief'' and asks another other Grappa nodes (``victim'') for
work. If the victim has tasks to spare in its public queue, they will be
transferred to the thief's public queue.

Most prior work on work stealing assumes that a processor has one worker
thread and so it only steals when utilization would go to zero. Grappa,
on the other hand, relies on having many worker threads per core. The
core would be fully-utilized if there is always something on the ready
queue. Even if there are many active tasks, if they are all suspended
for long-latency network requests, then the core is underutilized. So,
there is a choice: should the core use some of this idle time to perform
work stealing or should it just wait as it is possible that local tasks
will create more work. We choose to have a core initiate a steal when
all of the following conditions hold: no workers are ready to run, the
unstarted task queues are empty, and there are no outstanding steal
requests. Having only one outstanding steal request throttles steals to
prevent flooding the network, but other scalable quieting mechanisms
would be possible, such as a voting
tree\cite{scalableWorkStealingOrCilk98} or lifelines \cite{lifelines}.
Termination detection is not built into the Grappa task scheduler,
rather, it is considered a programming error for the program to exit
without syncing all tasks.

\TODO{Someone check this paragraph to make sure it is true. :-)  -Mark}

When a grappa node decides to issue a steal request it chooses the
victim node at random.  It also requests a block of tasks at once.  If
this block exceeds the amount of stealable tasks on the victim node,
half of the stealable tasks are stolen instead.  While we did study the
number of tasks to steal at once, we found performance was maximized
with \checkme{128} tasks with little sensitivity around this number,
hence we omit results for this from Section~\ref{sec:results} and simply
use \checkme{128} as the parameter for all experiments.


\subsection{Memory}

\TODO{Is there an off by 1 error in here?  Need to read the AMD64 docs carefully to see if we can sign extend the top bit of the 48 bits of real virtual address space, or if we need to maintain bit 49.  -M}

Grappa implements a software distributed shared memory. All global
addresses in Grappa are 64-bit values. Each byte of global memory is
associated with a {\em home core}; this core is responsible for
allocating and modifying that memory. Since today's commodity processors
support a 48-bit virtual address space \cite{AMD64}, Grappa uses the
spare bits to distinguish which address space a global address belongs
to, and to represent the home core of a global address.

Global memory in Grappa is exposed through two distinct, non-overlapping
address spaces; \textbf{2D global addresses}, optimized for global
access to node-local data, and \textbf{linear global addresses},
optimized for high aggregate random access bandwidth to low locality
data.


\begin{figure}[t]
\begin{center}
  \includegraphics[width=0.95\columnwidth]{figs/memory-structure}
\begin{minipage}{0.95\columnwidth}
  \caption{\label{fig:memory-structure} Grappa memory structure}
\end{minipage}
\vspace{-3ex}
\end{center}
\end{figure}

\paragraph{2D global addresses} The first global address space is the {\em
two-dimensional} address space. This uses a traditional PGAS addressing model,
where each address is a tuple of a rank in the job (or global process ID) and
an address in that process. The lower 48 bits of the address hold a virtual
address in the process. The top bit is set to indicates that the reference is
a 2D address. This leaves 15 bits for network endpoint ID, which limits our
scalability to $2^{15}$ endpoints.

Any node-local data can be made accessible by other nodes in the system
by wrapping the address and node ID into a 2D global address. This
address can then be accessed with a delegate and can also be cached by
other nodes. At the destination the address is converted into a
canonical x86 address by replacing the upper bits with the sign-extended
upper bit of the virtual address. 2D addresses may refer to memory
allocated from a single processes' heap or from a task's stack.
Figure~\ref{fig:memory-structure} shows how 2D and linear addresses can
refer to other cores' memory.

\paragraph{Linear global addresses} The second global address space is
the {\em linear} address space, so named because nearby addresses can be
treated as adjacent even if they are stored on different nodes.
Addresses in the linear address space are partitioned across the cluster
in a block-cyclic fashion. Choosing the block size involves trading off
sequential bandwidth against aggregate random access bandwidth. Smaller
block sizes help spread data across all the memory controllers in the
cluster, but larger block sizes allow the locality-optimized memory
controllers to provide increased sequential bandwidth. The block size,
which is configurable, is typically set to 64 bytes, or the size of a
single hardware cache line, in order to exploit spatial locality when
available. 

All linear addresses refer to data allocated from a global heap. The
heap metadata is stored on a single node. Currently all heap operations
serialize through this node; while this has been sufficient for our
benchmarks, in the future Grappa will provide parallel performance
through combining~\cite{MAMA,flatcombining}.

\TODO{Add a figure that depicts both forms of global addresses.}

All global memory operations are implemented with active messages.
Requesters construct a descriptor holding the current task ID and space
for a response, issue their request active message to the node given by
the request address, and suspend themselves. When the request arrives at
the address' home node, the operation is performed and a response active
message is sent. When the response arrives at the requesting node, the
descriptor is filled in with the results of the operation and the
requesting task is placed back on the ready queue.

\paragraph{Explicit caching.} Under the hood, Grappa performs the
mechanics of gathering chunks of data from multiple system nodes and
presenting a conventional appearing linear block of memory as a local
pointer into a cache. The strategy employed is to issue all the
constituent requests of a cache access request and then yield until all
responses have occurred.  Currently, Grappa caches are \emph{not}
coherent, requiring the programmer to maintain consistent access to
data.  Future work will develop a software directory based coherence
scheme to simplify consistent access to global data.

\paragraph{Delegates} Delegate operations are used for low-locality
accesses and synchronization. Their modifications are always executed at
the home core of their address, and while arbitrary memory operations
can be delegated, we restrict the use of delegate operations in three
ways to make them more useful for synchronization. First, we limit each
task to one outstanding delegate operation to avoid the possibility of
reordering in the network. Second, we limit delegate operations to
operate on objects in the 2D address space or objects that fit in a
single block of the linear address space so they can be satisfied with a
single network request. Finally, no context switches are allowed while
the data is being modified. Given these restrictions, we can ensure that
delegate operations for the same address from multiple requesters are
always serialized through a single core in the system, providing atomic
semantics without using atomic operations.


\subsection{Network Aggregation}

In order to mitigate the low message injection rate limits of commodity
networks, Grappa's communication stack has two layers: one for
user-level messages and one for network-level messages. 

At the upper layer, Grappa implements asynchronous active messages
\cite{vonEicken92}. Each message consists of a function pointer, an
optional argument payload, and an optional data payload. When a task
sends a message, the message is copied to a send queue associated with
the message's destination and the task continues execution.

Grappa's lower networking layer aggregates the upper layer's messages
to improve performance. Commodity networks including infiniband
achieves their peak bisection bandwidth \emph{only} when the packet
sizes are relatively large---on the order of multiple kilobytes. The
reason for this discrepancy is the combination of overheads associated
with handling each packet (in terms of bytes that form the actual
packet, processing time at the card and processing on the
CPU within the driver stack). Our measurements confirm manufacturers
published data~\cite{infinibandbandwidth}, that with this packet size
the bisection bandwidth is only a small fraction, less than
~\checkme{5\%} of the peak bisection bandwidth.

In our experiments the vast majority of requests were smaller than
\checkme{255} bytes \TODO{more precise? median?}, far too small to make
efficient use of the network. To make the best use of the network, we
must convert our small messages into large ones
(Section~\ref{sec:results-aggregation}).  When a task sends a message,
it is not immediately sent, but rather placed in a queue specific to the
destination.

There are three situations in which a queue of aggregated messages is
sent. First, each queue has a message size threshold, chosen to give
reasonable network performance (\checkme{2048 bytes}). If the size in
bytes of a queue is above the threshold, the contents of the queue are
sent immediately. Second, each queue has a wait time threshold
(\checkme{200ms}). If the oldest message in a queue has been waiting
longer than this threshold, the contents of the queue are sent
immediately, even if the queue size is lower than the message size
threshold.  Third, queues may be explicitly flushed in situations where
the programmer wants to minimize the latency of a message at the cost of
bandwidth utilization.

The network layer is serviced by polling. Periodically when a context
switch occurs, the Grappa scheduler switches to the network polling
thread. This thread has three responsibilities. First, it polls the
lower-level network layer to ensure it makes progress. Second, it
deaggregates received messages and executes active message
handlers. Third, it checks to see if any aggregation queues have
messages that have been waiting longer than the threshold; if so, it
sends them.

Underneath the aggregation layer, Grappa uses the \gasnet~communication
library~\cite{gasnet} to actually move data. All interprocess
communication, whether on or off a cluster node, is handled by the
\gasnet~library. \gasnet~is able to take advantage of many communication
mechanisms, including ethernet and infiniband between nodes, as well as
shared memory within a node.

Some networks provide access to a remote machine's memory directly. This
would seem to be a good fit for a programming model focused on global
shared memory, but in fact we do not use it. In our experiments, we
found that RDMA operations are subject to the same message rate
limitations as all other messages on these cards, and thus using raw
RDMA operations for our small messages would make inefficient use of
bandwidth. Instead, we implement remote memory operations with active
messages. A byproduct of this design decision is that Grappa is not
limited to RDMA-capable networks.

