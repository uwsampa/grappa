\section{Metholody} \label{sec:method}

To explore the performance of the Grappa runtime we have implemented three algorithms: breadth first search, betweenness centrality, and unbalanced tree search.  These algorithms were implemented for the Cray XMT (our baseline) and for Grappa.  Performance results for Grappa were obtained on a 144 node cluster.  Individual nodes of this cluster contain the hardware depicted in Table~\ref{table:grappanode}.  Nodes are interconnected with both 10G Ethernet and Infiniband.  For all but startup and configuration, they are configured to communicate over the Infiniband network for these experiments.

The metric we use is algorithmic time, which means startup and loading of the data structure (from disk) is not included in the measurement.  Data is collected on real systems, which means minor variations in runtime exist from run to run.  The average of multiple runs are used, and where appropriate, confidence in the result is reported (error bars, standard deviation, etc).

One question that must be answered when making a comparison between the Cray XMT and a typical HPC cluster is what is a fair comparison -- these systems are quite different.  Three options immediately come to mind: equal number of processing cores, equal number of network interfaces, and equal dollars.  We discounted the last option fairly quickly, because it isn't a lasting data-point and, as will be evident later, would be hugely unfair to the XMT.  The first option, cores, has some merit, but Grappa is designed for applications that have no locality in their computation.  This means almost all of their memory accesses are remote.  Hence, the factor that limits their performance is not processing, but communicating.  Hence, we have chosen the middle option, network interfaces as the way to normalize across the XMT and our cluster.  Each processor in the XMT system has its own network interface to access shared memory.  In the HPC cluster we use, each processor (which contains 12 cores), has a single Infiniband interface.  Hence, for our results we scale up XMT processors one for one with full system nodes.

\subsection{Applications}
%Traversal of an unbalanced, unpartitioned tree captures the foundations for implementing most forms of irregular parallelism. 
%Recursion is found in every irregular divide and conquer algorithm. Loops, even regular ones, must be dynamically scheduled to get good load balance when run on a large, non-uniform system. Nested parallelism involves fine-grain creation of dynamic amounts of work. Dataflow..

\paragraph{Unbalanced tree search (UTS)}, is a benchmark for evaluating the programmability and performance of systems for parallel applications that require dynamic load balancing \cite{Olivier:uts2006}. It involves traversing an unbalanced implicit tree: at each vertex, its number of children is sampled from some probability distribution, and this number of new nodes are added to a work queue to be visited. While this benchmark captures irregular, dynamic parallelism, we want to evaluate performance on algorithms that touch existing data. We augment UTS by using the traversal to create a large tree in memory, and then we traverse the in-memory tree. We call this benchmark UTS-mem, and the timed portion is this traversal of the in-memory tree. The assumption is that in general the size of each subtree is unknown until that subtree has actually been traversed.

%todo: say tree explore is same as a search except we are additionally needing to synchronize on all vertices visited

The Grappa version of the in-memory tree search uses the asynchronous parallel for loop over a visited vertex's children list. A larger threshold can be set to take advantage of this locality by processing more child pointers sequentially in a single task.

\comment{The Grappa version of tree search UTS-mem

\lstset{language=C++,
       basicstyle=\footnotesize,
       tabsize=2}
\begin{lstlisting}
// base of shared vertex_t[]
GlobalAddress<vertex_t> Vertices;        

// base of shared int64_t[]
GlobalAddress<int64_t> ChildrenPointers;  

void search_vertex( id ) {
  GlobalAddress<vertex_t> v_addr = Vertices + id;
  Incoherent<vertex_t>::RO v( v_add, 1 );

  // start index of my children pointers
  childIndex = v->childIndex;    

  // how many children I have
  numChildren = v->numChildren;  

  // parallel loop over the child list for this vertex
  parallel_for(fn=&search_children, start=childIndex, iters=numChildren);
}

void search_children( start, iters ) {
  // take advantage of spatial locality in the array of children
  GlobalAddress<int64_t> child_base_addr = ChildrenPointers + start;
  Incoherent<int64_t>::RO childIds( child_base_addr, iters );

  // spawn a task to visit each child
  for (i = 0..iters) {
    SoftXMT_publicTask( fn=&search_vertex, id=childIds[i] );
  }
}
\end{lstlisting}
}

\paragraph{Breadth-first-search (BFS)} This is the primary kernel for the Graph500 benchmark and is what currently determines the ranking of machines on the Graph500 list~\cite{graph500list}. As a whole, the Graph500 benchmark suite is designed to bring the focus of system design on data-intensive workloads, particularly large-scale graph analysis problems, that are important among cybersecurity, informatics, and network understanding workloads. Graph500's definition of BFS of a graph consists of computing a search tree containing parent nodes for each traversed vertex during the search. While this is a relatively simple problem to solve, it exercises the random-access and fine-grained synchronization capabilities of a system as well as being a primitive in many other graph algorithms. Performance is measured in \emph{traversed edges per second} (TEPS), where the number of edges is the edges making up the generated BFS tree. One of the reference implementations of Graph500 BFS is for the XMT. We compare this against a straightforward Grappa implementation that algorithmically matches the XMT reference code to allow us to fairly compare the execution systems. We do not employ algorithmic improvements beyond the reference code, though there are many \cite{Beamer:Graph500,Yoo:FixedPointGraph500}.


