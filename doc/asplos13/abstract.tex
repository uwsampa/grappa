Grappa is a latency tolerant runtime system for commodity multinode multicore
platforms that addresses the challenges posed in scaling irregular parallel
applications to large datasets. These applications are of growing importance
as interactive analysis of large unstructured data sets emerges as a
commercial and government priority. Grappa serves both as a C++ user library
and as a foundation upon which higher level languages can be developed or
adapted to support development of irregular parallel applications. Grappa
tolerates the delays to distant memory by multiplexing hundreds to thousands
of lightweight tasks to each processor core; balances load via fine-grained
workstealing; and provides efficient synchronization and consistency across
its single shared address space. We present a detailed description
of the Grappa system, programming examples using the library interface, and
compare performance for several irregular benchmarks to the Cray XMT, a custom
system used to target the real time graph analytics market.