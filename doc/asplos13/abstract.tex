Grappa is a latency tolerant runtime system for commodity clusters of
multicore computers, its goal is to provide scalable performance for irregular
parallel applications. These applications are of growing importance as
interactive analysis of large data sets emerges as a commercial and government
priority, typically in the form of graph processing. Grappa serves both as a
C++ user library and as a foundation upon which higher level languages can be
developed or adapted to support development of irregular parallel
applications. Grappa tolerates the delays to distant memory by multiplexing
thousands of lightweight tasks to each processor core; balances load via
fine-grained distributed work-stealing; takes full advantage of network
characteristics by aggregating smaller requests into large ones; and provides
efficient synchronization and remote operations across its single shared
address space. We present a detailed description of the Grappa system,
programming examples using the library interface, and compare performance for
several irregular benchmarks to the Cray XMT, a custom system used to target
the real time graph analytics market.