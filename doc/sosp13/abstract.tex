\Grappa is a runtime system for commodity clusters of multicore computers that
presents a massively parallel, single address space abstraction to
applications. \Grappa's purpose is to enable scalable performance of irregular
parallel applications, such as  branch and bound optimization and SPICE circuit simulation, and graph processing. Poor data locality,
imbalanced parallel work and complex communication patterns make scaling these
applications difficult.

\Grappa serves both as a C++ user library and as a foundation for higher level
languages. \Grappa tolerates delays to remote memory by multiplexing thousands
of lightweight workers to each processor core; balances load via fine-grained
distributed work-stealing; increases communication throughput by aggregating
smaller data requests into large ones; and provides efficient synchronization
and remote operations. We present a detailed description of the \Grappa system
and performance comparisons on several irregular benchmarks to hand-optimized
MPI code and to the Cray XMT, a custom system used to target the real time
graph analytics market.
