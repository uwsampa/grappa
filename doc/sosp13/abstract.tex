\Grappa is a runtime system for commodity clusters of multicore computers that
presents a massively parallel, single address space abstraction to
applications. \Grappa's purpose is scalable performance for irregular
parallel applications, such as graph processing. Poor data locality, imbalanced parallel
work and complex communication patterns make these applications difficult to scale.

\Grappa serves both as a C++ user library and as a foundation upon which
higher level languages can be developed or adapted. \Grappa tolerates delays
to remote memory by multiplexing thousands of lightweight workers
to each processor core; balances load via fine-grained distributed
work-stealing; increases communication throughput by aggregating
smaller data requests into large ones; and provides efficient synchronization
and remote operations. We present a detailed description of the \Grappa
system and
%programming examples using the library interface,
performance comparisons on several irregular benchmarks to hand-optimized MPI code and to
the Cray XMT, a custom system used to target the real time graph analytics
market.
