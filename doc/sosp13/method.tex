\section{Methodology} \label{sec:method}

\TODO{Need to rewrite once our evaluations story is solid. We need to talk about MPI and why that comparison makes sense.}

To explore the performance of the Grappa runtime we have implemented
three algorithms: breadth first search, betweenness centrality, and
unbalanced tree search.  These algorithms were implemented for the
Cray XMT (our baseline) and for Grappa.  The metric we use is
algorithmic time, which means startup and loading of the data
structure (from disk) is not included in the measurement.  Data is
collected on real systems, which means minor variations in runtime
exist from run to run.  The average of multiple runs are used, and
where appropriate, confidence in the result is reported.

One question that must be answered when making a comparison between the
Cray XMT and a typical HPC cluster is what is a fair comparison -- these
systems are quite different.  Three options immediately come to mind:
equal number of processing cores, equal number of network interfaces,
and equal dollars.  We discounted the last option fairly quickly,
because it isn't a lasting data-point -- the cost of hardware shifts
over time, skewing the interpretation of results.  The first option,
cores, has some merit, but Grappa is designed for applications that have
no locality in their computation.  This means almost all of their memory
accesses are remote.  The factor that limits their performance is not
processing, but communicating.  Hence, we have chosen the middle option,
network interfaces as the way to normalize across the XMT and our
cluster.  Each processor in the XMT system has its own network interface
to access shared memory.  In the HPC cluster we use, each processor
(which contains up to 32~cores, although we only use 6 in our
experiments), has a single infiniband interface.  Hence, for our results
we scale up XMT processors one for one with full system nodes.


\subsection{Systems}

For measurements, we run Grappa on a 144-node cluster of AMD Interlagos
processors. Nodes have 32-core (every pair share a floating-point unit)
2.1-GHz processors, 64GB of memory, and 40Gb Mellanox ConnectX-2
infiniband network card.   The cluster uses a QLogic infiniband switch.
We configure the nodes to have 32 1-GB hugepages to minimize TLB misses
for the random access patterns we expect from irregular applications.

We compare to the MTA using a 128-node Cray XMT (3rd generation MTA).
Each node consists of a 500-MHz MTA Threadstorm multithreaded processor
that supports 128~streams. The machine uses Cray's proprietary SeaStar2
interconnection network.

\subsection{Applications}

We have used three benchmarks to explore the performance of Grappa:

\paragraph{Unbalanced tree search in-memory (UTS-Mem)} Unbalanced Tree
Search (UTS) is a benchmark for evaluating the programmability and
performance of systems for parallel applications that require dynamic
load balancing~\cite{UTS}. It involves traversing an
unbalanced implicit tree: at each vertex, its number of children is
sampled from some probability distribution, and this number of new nodes
are added to a work queue to be visited. While this benchmark captures
irregular, dynamic \emph{computation,} we actually want to evaluate
performance of algorithms with irregular \emph{memory\/} access patterns.
Thus we augment UTS by using the existing traversal code to create a
large tree in memory, and then we traverse the in-memory tree. We call
this benchmark UTS-Mem, and the timed portion is this traversal of the
in-memory tree. This in-memory traversal has no knowledge of the
tree structure beforehand.  The Grappa version of the in-memory tree
search uses the asynchronous parallel for loop over a visited vertex's
children list.

\comment{The Grappa version of tree search UTS-mem

\lstset{language=C++,
       basicstyle=\footnotesize,
       tabsize=2}
\begin{lstlisting}
// base of shared vertex_t[]
GlobalAddress<vertex_t> Vertices;        

// base of shared int64_t[]
GlobalAddress<int64_t> ChildrenPointers;  

void search_vertex( id ) {
  GlobalAddress<vertex_t> v_addr = Vertices + id;
  Incoherent<vertex_t>::RO v( v_add, 1 );

  // start index of my children pointers
  childIndex = v->childIndex;    

  // how many children I have
  numChildren = v->numChildren;  

  // parallel loop over the child list for this vertex
  parallel_for(fn=&search_children, start=childIndex, iters=numChildren);
}

void search_children( start, iters ) {
  // take advantage of spatial locality in the array of children
  GlobalAddress<int64_t> child_base_addr = ChildrenPointers + start;
  Incoherent<int64_t>::RO childIds( child_base_addr, iters );

  // spawn a task to visit each child
  for (i = 0..iters) {
    SoftXMT_publicTask( fn=&search_vertex, id=childIds[i] );
  }
}
\end{lstlisting}
}

\paragraph{Breadth-first-search (BFS)}
This is the primary kernel for the Graph500 benchmark and is what
currently determines the ranking of machines on the Graph500
list~\cite{graph500list}. As a whole, the Graph500 benchmark suite is
designed to bring the focus of system design on data-intensive
workloads, particularly large-scale graph analysis problems, that are
important among cybersecurity, informatics, and network understanding
workloads. The BFS benchmark builds a search tree containing parent
nodes for each traversed vertex during the search.  While this is a
relatively simple problem to solve, it exercises the random-access and
fine-grained synchronization capabilities of a system as well as being
a primitive in many other graph algorithms. Performance is measured in
\emph{traversed edges per second\/} (TEPS), where the number of edges is
the edges making up the generated BFS tree. One of the reference
implementations of Graph500 BFS is for the XMT; this code fails to
scale past 16 XMT processors because it does not expose enough
parallelism, so we modified the code to use a recursive loop
decomposition similar to Grappa's. We compare this modified version
against a straightforward Grappa implementation. We do not employ
algorithmic improvements, though there are
many~\cite{Beamer:Graph500,Yoo:FixedPointGraph500}.

\paragraph{Betweenness Centrality}
An important measure of the importance of particular vertices in a
network is betweenness centrality
(BC)~\cite{freeman1979centrality}. By this measure, the ``importance''
of each vertex is computed by finding the fraction of shortest paths
that pass through it, which can optionally be approximated by only
computing shortest paths using a subset of the vertices as starting
points. BC can be useful for understanding which vertices may have the
greatest impact, so in social networks this could be the primary
person linking two communities. Because it requires multiple
breadth-first traversals across the entire graph and in reverse, on
power-law degree graphs, this algorithm exercises random accesses
rate, and load balancing. It also requires fine-grained
synchronization on updates to vertex centrality values because
multiple paths will update the same vertex. BC is a kernel in the
DARPA High Performance Computing Systems (HPCS) Scalable Synthetic
Compact Applications graph analysis (SSCA\#2)
benchmark~\cite{ssca2}. Performance for BC is also measured in TEPS,
where the number of traversed edges is the total number of edges in
the graph multiplied by the number of random starting vertices used in
the approximate computation. We use the XMT implementation of BC
implemented as part of the GraphCT~\cite{GraphCT} library and a
comparable Grappa implementation, both of which use the parallel BC
algorithm developed by Bader et al.~\cite{bader:bc}.

