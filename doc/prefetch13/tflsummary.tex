\section{TFL Argument}

\begin{enumerate}

\item Previous work assumed that in saturation, ie when $v$, the
    number of threads is enough to cover the latency of a remote
    access, the processor utilization is $U=R/(R+S)$, where $R$ is the
    time between remote accesses and $S$ is synchronization time.
    
\item \cite{culler:93} argued that because there is increasing cost to
    switch between more threads, $S$ must be an increasing function of
    $v$, and the execution of the processor may slow down due to larger
    $v$, leading to $U= \frac{1}{C_v} * \frac{R}{R+S_v}$, where $S_v$ is increasing
    with $v$ and $C_v$, the increased program execution time, is increasing with
    $v$.

\item While this analysis is sound, $S_v$ and $C_v$ are not fundamentally
    increasing due to latency to top level but rather bandwidth to
    thread data. (Not entirely clear to me that they ruled out
    bandwidth or not, hard to tell what they were thinking, see
    ``physics and dollars")

\item This changes how we think about this limit, if we can increase
    bandwidth (cost-based, although I’m not sure I fully believe this)
    then the number of threads can be much higher than if we are
    limited by latency (also cost-based but primarily physics). 

\end{enumerate}
