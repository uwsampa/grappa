%\documentclass{acm_proc_article-sp}
%\documentclass[10pt,twocolumn]{article}
\documentclass[pageno, 10pt]{asplos14-template/jpaper}

% NOTE: SOSP submission guidelines say up to 13 pages of technical content, which is everything except references. References do NOT count towards page liites. And it is support 10 pt. SOSP is *very* picky about font size. 

%replace XXX with the submission number you are given from the ASPLOS submission site.
\newcommand{\asplossubmissionnumber}{217}

%%% page size formatting %%% 
%\usepackage[margin=1in]{geometry}

\setlength{\columnsep}{0.25in}

%%% packages %%%
\usepackage[normalem]{ulem}
\usepackage{color}
\usepackage[usenames,dvipsnames]{xcolor}

\usepackage{graphicx}
\usepackage{float}
\usepackage{url}
\usepackage{times}
\usepackage{xspace}

%%% code listings %%%
\usepackage{listings}

\usepackage{inconsolata}

\lstset{
  emph={GlobalAddress,Grappa,ConditionVariable,Mutex,T,key_t,node_t,int64_t,Promise}, emphstyle={\color{NavyBlue}},
  emph={[2]delegate, write, acquire, release, core, call, signal, push, pop, sync, block, memcpy}, emphstyle={[2]\color{PineGreen}}
}
\lstdefinestyle{grappa}{ 
  belowcaptionskip=1\baselineskip,
  breaklines=true,
  %frame=L,  % frame around code
  xleftmargin=\parindent,
  language=C++,
  showstringspaces=false,
  %basicstyle=\tiny,
  basicstyle=\footnotesize\ttfamily,
  keywordstyle=\color{Purple},
  commentstyle=\itshape\color{Gray},
  identifierstyle=\color{Black},
  stringstyle=\color{RedOrange},
  morekeywords={override},
  numbers=none,
}
\lstset{escapechar=@,style=grappa}

%%% utilities %%%
\newcommand{\TODO}[1]{\textcolor{red}{{\bf TODO:} #1}}
\newcommand{\checkme}[1]{\textcolor{red}{\textbf{#1}}}
\newcommand{\grappacomment}[1]{}

%%% literal terms %%%
\newcommand{\gasnet}{GASNet}
\newcommand{\Grappa}{Manta\xspace}

% runtime parameter names 
\newcommand{\flushtimeout}{\emph{network\_flush\_timeout}}
\newcommand{\asyncforthr}{\emph{async\_parallel\_for\_thresold}}


\usepackage{microtype}
\begin{document}



\title{\Large \Grappa: A Latency-Tolerant Runtime for Large-Scale Irregular Applications}

% from the submissions guidelines: On the front page, in place of the authors'
% names, the paper should indicate: the paper ID number assigned during the
% paper registration process and the total number of pages in the submission.a

\author{Paper ID: {\bf 51}, {\bf 14} pages.}
\date{}
% \authorinfo{Jacob Nelson$^{\dagger}$,
%   Brandon Holt$^{\dagger}$,
%   Brandon Myers$^{\dagger}$,
%   Preston Briggs$^{\dagger}$,
%   Luis Ceze$^{\dagger}$,
%   Simon Kahan$^{{\dagger \ddagger}}$,
%   Mark Oskin$^{\dagger}$
% }{\textdagger University of Washington, \textdaggerdbl Pacific Northwest National Laboratory}{
%   \{nelson, bholt, bdmyers, preston, luisceze, skahan, oskin\}@cs.washington.edu}


\maketitle
\begin{abstract}
Grappa is a runtime system for commodity clusters of multicore computers that
presents a massively parallel, single address space abstraction to
applications. Grappa's goal is to provide scalable performance for irregular
parallel applications, such as graph processing. These applications are
challenging because they exhibit little data locality, imbalanced parallel
work and complex communication patterns.

Grappa serves both as a C++ user library and as a foundation upon which higher
level languages can be developed or adapted. Grappa tolerates delays to remote
memory by multiplexing thousands of lightweight tasks to each processor core;
balances load via fine-grained distributed work-stealing; takes full advantage
of network characteristics by aggregating smaller data requests into large
ones; and provides efficient synchronization and remote operations. We present
a detailed description of the Grappa system, programming examples using the
library interface, and compare performance for several irregular benchmarks to
hand-optimized MPI code and the Cray XMT, a custom system used to target the
real time graph analytics market.

\end{abstract}

\section{Introduction} \label{sec:intro}

Irregular applications exhibit workloads, dependences, and memory accesses
that are highly sensitive to input. Classic examples of such applications
include branch and bound optimization, SPICE circuit simulation, and car crash
analysis. Important contemporary examples include processing large graphs in
the business, national security, and social network computing domains. For
these emerging applications, reasonable response time -- given the sheer
amount of data -- requires large multinode systems. The most broadly available
multinode systems are those built from x86 compute nodes interconnected via
ethernet or InfiniBand. However, scalable performance of irregular
applications on these systems is elusive for two reasons:

\vspace{0.5ex}
\noindent{\bf Poor data locality and frequent communication.} Data reference
patterns of irregular applications are unpredictable and tend to be spread
across the entire system. This results in frequent requests for remote data.
Caches are of little assistance because of low data re-use and spatial
locality. Prefetching is ineffective because request locations are not known
early enough. Data-parallel frameworks such as
MapReduce~\cite{mapreduce:osdi04} are also inappropriate because they rely on
being able to partition data and regular communication patterns. As a
consequence, irregular applications require frequent communication of
typically small pieces of data. In contrast, commodity networks are designed
for large packets and have just a fraction of their peak bandwidth when moving
small packets.

\vspace{0.5ex} \noindent{\bf High network communication latency.} The
performance challenges of frequent communication is exacerbated by the network
latency relative to processor performance. Latency of commodity networks runs
anywhere from a few to hundreds of microseconds, which is much more than
superscalar execution (even with sophisticated memory hierarchies) can
tolerate. Since irregular application tasks encounter remote references
dynamically during execution and must resolve them before making further
progress, stalls are frequent and lead to severely underutilized compute
resources.

While some irregular applications can be manually restructured to better
exploit locality, aggregate requests to increase network message size, and
manage the additional challenges of load balance and synchronization, the
effort required to do so is formidable and involves knowledge and skills
pertaining to distributed systems far beyond those of most application
programmers. Luckily, many of the important irregular applications (e.g.,
graph processing, our focus in this paper) naturally offer large amounts of
concurrency. This immediately suggests taking advantage of concurrency to
tolerate the latency of data movement by overlapping computation with
communication.

The fully custom Tera MTA-2~\cite{tera:mta1} system is a classic example of
supporting irregular applications by using concurrency to hide latencies. It
had a large distributed shared memory with no caches. On every clock cycle,
each processor would execute a ready instruction chosen from one of its 128
hardware thread contexts, a sufficient number to fully hide memory access
latency. The network was designed with a single-word injection rate that
matched the processor clock frequency and sufficient bandwidth to sustain a
reference from every processor on every clock cycle. Unfortunately, the MTA
was not general enough nor cost-effective. The Cray XMT approximates the Tera
MTA-2, reducing its cost but not overcoming its narrow range of applicability.

We believe we can support irregular applications with good performance and
cost-effectiveness with commodity hardware for two main reasons. First,
commodity multicore processors have become extremely fast and cheap. So with
careful software engineering we can multiplex hundred of thousands of
workers into a single core. Second, commodity networks offer high
bandwidth as long as messages are large enough. So with enough data requests
in flight simultaneously, the system can aggregate small messages into large
enough ones.

In this paper we introduce \Grappa, a software runtime system that allows a
commodity cluster of x86-based nodes connected via an InfiniBand network to be
programmed as if it were a single, large, shared-memory NUMA (non-uniform
memory access) machine with scalable performance for irregular applications.
\Grappa is designed to smooth over some of the performance discontinuities in
commodity hardware, giving good performance when there is little locality to
be exploited while allowing the programmer to exploit it when it is available.

\Grappa leverages as much freely available and commodity infrastructure as
possible. We use unmodified Linux for the operating system and an
off-the-shelf user-mode InfiniBand device driver stack~\cite{OFED}. MPI is
used for process setup and tear down. GASNet~\cite{gasnet} is used as the
underlying mechanism for remote memory reads and writes using active message
invocations. \Grappa adds three main software components: (1) a
\emph{lightweight tasking\/} layer that supports a context switch in as little
as 38ns and distributed global load balancing; (2) a \emph{distributed shared
memory\/} layer that supports normal access operations such as \emph{read\/}
and \emph{write\/} as well as synchronizing operations such as
\emph{fetch-and-add\/}~\cite{fetchandadd}; and (3) a \emph{communication\/}
layer that combines short messages to achieve peak bandwidth on commodity
networks. As we will show later, \Grappa can tolerate latencies way beyond
that of the network. Therefore, \Grappa can afford to \emph{trade latency for
throughput}. By {\em increasing\/} latency in key components of the system we
are able to: increase effective random access memory bandwidth by delaying and
aggregating messages; increase synchronization rate by delegating atomic
operations to gatekeeper cores, even when referencing node-local global data;
and improve load balance by tolerating the delays incurred when work-stealing.

Our evaluation of Grappa shows that it runs several irregular application
kernels (e.g., graph traversal and sort) very efficiently on a commodity
cluster. Our yardstick for comparison is the XMT hardware itself. Using the
same number of network interfaces, \Grappa is in the same ballpark as the XMT:
For unbalanced tree search, Grappa is over \TODO{3X} faster and shows greatly
improved scalability; conversely, for breadth first search and betweenness
centrality Grappa is 2.5X slower. In Section~\ref{sec:evaluation} we explore
the factors that underpin this performance. Most importantly, however, for
significantly less real world cost, users can \emph{add\/} significantly more
processors to a commodity cluster than an XMT machine and use Grappa to
achieve scalable performance. \TODO{revise paragraph after eval, including MPI
and possibly programming model}




\section{Grappa Overview}

\begin{figure}[t]
\begin{center}
  \includegraphics[width=0.95\columnwidth]{figs/system-overview}
\begin{minipage}{0.95\columnwidth}
  \caption{\label{fig:grappa} Grappa system overview}
\end{minipage}
\vspace{-3ex}
\end{center}
\end{figure}


Grappa (Figure~\ref{fig:grappa}) has three main software components:
\begin{description}

\item [Tasking system.] Our tasking system supports lightweight
multithreading to tolerate communication latency and global distributed
workstealing (i.e., tasks can be stolen from any node in the system), which
provides automated load balancing.

\item[Distributed shared memory.] Our DSM system provides support for
fine-grain access to data anywhere in the system. It supports synchronization
operations on global data, explicit local caching of any memory in the system,
and support for operation on remote data (delegating operations to home node).
By tight integration with the tasking system and the
communication layer, our DSM system offers high aggregate random
access bandwidth for accessing remote data.

\item[Communication layer.] As discussed earlier, modern commodity networks
support high bandwidth only for large messages. Since irregular applications
tend to need frequent communication of small requests, the main goal of our
communication layer is to aggregate small messages into large ones to better
exploit what the network can offer. It is largely invisible to the application
programmer.

% software developer but helps to improve network performance when applications read and write only small pieces of data.

\end{description}

\paragraph{Exploiting Latency Tolerance} As we will show later, Grappa can tolerate latencies way beyond that of the network. Therefore, Grappa can afford to \emph{trade latency for throughput}: by {\em increasing} latency in key components of the system we are able to increase our aggregate random access bandwidth (by delaying and aggregating messages), our synchronization bandwidth (by delegating operations to remote nodes), and our ability to improve load imbalance (work stealing increases latency).

% \TODO{check to see we actually explain these three somewhere.  Ie, that aggregating increases throughput and latency, that delegating synchronization increases the rate and the latency at which we can eg atomically increment, and that stealing work increases the latency of some tasks (eg when their data is largely at the originating node) but provides greater throughput overall.}

In the next three sections we describe both Grappa's main capabilities and how they are implemented.



\section{Tasking System}

Below we discuss the implementation of our task management support and then
describe how applications expose parallelism to the \Grappa runtime.

\subsection{Task Support Implementation}

The basic unit of execution in \Grappa is a {\em task}. When tasks are ready to
execute, they are mapped to a {\em worker}, which is akin to a user-level
thread. Each hardware core has a single operating system thread pinned to it.

\paragraph{Tasks} 
Tasks are specified by a ``functor'' object that holds both code to execute and initial state. The functor can be specified with a function pointer and explicit arguments, a C++ struct that overloads the parentheses operator, or a C++11 lambda. These objects, typically very small (on the order of 64 bytes), hold read-only values such as an iteration index and pointers to common data or synchronization objects. Task functors can be serialized and transported around the system, and eventually executed by a worker, as described next.

\paragraph{Workers} Workers execute application and system (e.g.,
communication) tasks. A worker is simply a collection of status bits and a
stack, allocated at a particular core. When a task is ready to execute it
is assigned to a worker, that executes the task functor on its own stack. 
Once a task is mapped to a worker it stays with that worker until it finishes.

\paragraph{Scheduling} During execution, a worker yields control of its core
whenever performing a long-latency operation, allowing the processor to
remain busy while waiting for the operation to complete. In addition, a
programmer can direct scheduling explicitly via the \Grappa API calls shown in
Figure~\ref{fig:scheduling}. To minimize yield overhead, the \Grappa scheduler
operates entirely in user-space and does little more than store state of one
worker and load that of another.

\begin{figure}[htbp]
  \begin{center}
	\begin{tabular}{l}
    \texttt{\scriptsize yield() } \\
      Yields core to scheduler, enqueuing caller to be \\ scheduled again soon \\
    \texttt{\scriptsize wake( task * $t$ ) } \\
      Enqueues some other task $t$ to be scheduled again soon \\
    \texttt{\scriptsize suspend() }  \\
      Yields core to scheduler, enqueuing caller only once \\ another task calls wake \\
	\end{tabular}
    \begin{minipage}{0.95\columnwidth}
      \caption{\label{fig:scheduling} \Grappa scheduling API } 
    \end{minipage}
  \end{center}
\end{figure}

Each core in a \Grappa system has its own independent scheduler. The scheduler
has a single FIFO queue of active workers ready to execute, the {\it
ready worker queue}. Each scheduler also has three queues of tasks
waiting to be assigned a worker:

\begin{itemize}

\item {\it deadline task queue}, a priority queue of tasks that are executed according to task-specific deadline constraints;

\item {\it private task queue}, a FIFO queue of tasks that must run on
this core and therefore is not subject to stealing;

\item {\it public task queue},  a LIFO queue of tasks that are
  waiting to be matched with workers. It is a local partition of a shared
  task pool.

\end{itemize}


Whenever a task yields or suspends, the scheduler makes a decision about what
to do next. First, any task in the deadline task queue who's deadline
constraint is met is chosen for execution. This queue manages high priority
system tasks, such as periodically servicing communication requests. Second,
the scheduler determines if any workers with running tasks are ready to
execute; if so, one is scheduled. Finally, if there are no workers ready to
run, but there are tasks waiting to be matched with workers, an idle worker is
woken (or a new worker is spawned), matched with a task, and scheduled.

\paragraph{Context switching} 
\Grappa context switches between workers non-preemptively. As with other
cooperative multithreading systems, we treat context switches as function
calls, saving and restoring only the callee-saved state as specified in the
x86-64 ABI~\cite{amd64:abi:2012}. This involves saving six general-purpose
64-bit registers and the stack pointer, as well as the 16-bit x87 floating
point control word and the SSE context/status register. Thus, the minimum
amount of state a cooperative context switch routine must save, according to
the ABI, is 62~bytes.

Since \Grappa keeps a very large number of active workers, their context data
will not fit in cache. By oversubscribing on the number of workers
beyond what is required for latency tolerance, the scheduler can
ensure there is always some number of context pointers in the ready
queue. This allows the scheduler to prefetch contexts into
cache using software prefetch instructions. The reason this works is because the size of the L1 cache is
sufficient to hold enough contexts to tolerate the latency to main
memory. Empirically we find that prefetching the fourth worker in the scheduling order is
sufficient. This prefetching makes context switching effectively free of cache misses even to hundreds
of thousands of threads. We provide an analysis of our context switch
performance in Section~\ref{eval:basic}.


\paragraph{Work stealing} 
When the scheduler finds no work to assign to its workers, it commences to
steal tasks from other cores using an asynchronous \texttt{call\_on} active
message. It chooses a victim at random until it finds one with a non-zero
amount of work in its public task queue. The scheduler steals half of the
tasks it finds at the victim. Work stealing is particularly interesting in
\Grappa since performance depends on having many active worker threads on each
core. Even if there are many active threads, if they are all suspended on
long-latency operations, then the core is underutilized.

\subsection{Expressing Parallelism}

\TODO{addressing reviewer2: make it clear that yes Grappa is a low
    level programming model, but it provides the mechanisms for
    implementing a variety of execution models. Further, in some cases
    like the parallel loops, we have allowed for may parallelism to
    give flexibility to the implementation.}
\Grappa programmers focus on expressing as much parallelism as possible
without concern for where it will execute. \Grappa then chooses where and when
to exploit this parallelism, scheduling as much work as is necessary on each
core to keep it busy in the presence of system latencies and task dependences.

\Grappa provides four methods for expressing parallelism, shown in
Figure~\ref{fig:expressing-parallelism}. First, when the programmer identifies
work that can be done in parallel, the work may be wrapped up in a function
and queued with its arguments for later execution using a \texttt{spawn}.
Second,
the programmer can invoke a parallel for loop with \texttt{parallel\_for}, provided that the trip count is
known at loop entry. The programmer specifies a function pointer along with
start and end indices and an optional threshold to control parallel overhead.
\Grappa does {\em recursive decomposition} of iterations, similar to Cilk's
cilk\_for construct~\cite {cilkforimplementation}, and TBB's {\tt
parallel\_for}~\cite{intel_tbb}. It generates a logarithmically-deep tree of
tasks, stopping to execute the loop body when the number of iterations is
below the required threshold. Third, a programmer may want to execute an active message; that is, to run a
small piece of code on a particular core in the system without waiting for
execution resources to be available. Custom synchronization primitives, for example, execute this way as a function executed on the core where the data
lives. \Grappa provides the \texttt{call\_on} call for this purpose.

\begin{figure}[htbp]
  \begin{center}
	\begin{tabular}{l}
    \texttt{\scriptsize spawn( Functor f )} \\
      Creates a new stealable task \\
    \texttt{\scriptsize parallel\_for( start, end, Functor iteration )} \\
      Executes iterations of a loop as stealable tasks that \\
      take the iteration index as an argument  \\
    \texttt{\scriptsize call\_on( core, Functor f )} \\ 
      Runs a limited function on a specific core without \\
      consuming \Grappa execution resources 
	\end{tabular}
    \begin{minipage}{0.95\columnwidth}
      \caption{\label{fig:expressing-parallelism} \Grappa API: expressing parallelism
      } % \vspace{-4ex}}
    \end{minipage}
    %\vspace{-3ex}
  \end{center}
\end{figure}

Figure~\ref{fig:sample} shows sample code using \Grappa for a parallel tree
search. The important aspect to note is that the code looks very similar to
what would be written for single shared-memory system, without any concern about data locality or communication.

\begin{figure}[htbp]
\begin{center}
\begin{scriptsize}
\begin{verbatim}
class node_t {
  key_t   key
  int64_t numChildren;
  GlobalAddress<node_t *> children;
};

void search(GlobalAddress<node_t *> node, key_t key) {
  if (node->key == key) {
    result = node;
    return;
  }
  parallel_for(0, node->numChildren, [=](int index) {
    spawnTask([=]{
      search(node->children[index], key);
    });
  });
}
\end{verbatim}
\end{scriptsize}

    \begin{minipage}{0.95\columnwidth}
      \caption{\label{fig:sample} Sample \Grappa code illustrating a parallel tree search similar to the unbalanced tree search benchmark we describe later.}
    \end{minipage}

\end{center}
\end{figure}



\section{Distributed Shared Memory}

Applications written for \Grappa utilize two forms of memory: local and
global. Local memory is local to a single core within a node in the system.
Accesses occur through conventional pointers. Applications use local accesses for a
number of things in \Grappa: the stack associated with a task, accesses to
localized global memory in caches (see below), and accesses to debugging
infrastructure that is local to each system node. Local pointers cannot access
memory on other cores, and are valid only on their home core.

Large data that is expected to be shared and accessed with low locality is
stored in \Grappa's global memory. All global data must be accessed through
calls into \Grappa's API, shown in Figure~\ref{fig:accessing-memory}.

\TODO{worth mentioning localized access for data-parallel stuff, along
    with reasoning about local vs remote accesses as in the PGAS world}

\paragraph{Global memory addressing} \Grappa provides two methods for storing
data in the global memory. The first is a distributed heap striped across all
the machines in the system in a block cyclic fashion. The
\texttt{global\_malloc} and \texttt{global\_free} calls are used to allocate
and deallocate memory in the global heap. Addresses to memory in the global
heap use \emph{linear addresses}. Choosing the block size involves trading off
sequential bandwidth against aggregate random access bandwidth. Smaller block
sizes help spread data across all the memory controllers in the cluster, but
larger block sizes allow the locality-optimized memory controllers to provide
increased sequential bandwidth. The block size, which is configurable, is
typically set to 64 bytes, or the size of a single hardware cache line, in
order to exploit spatial locality when available.

\Grappa also allows any local data on a core's stacks or heap to be exported
to the global address space to be made accessible to other cores across the
system. Addresses to global memory allocated in this way use \emph{2D global
addresses}. This uses a traditional PGAS (partitioned global address
space~\cite{upc:2005}) addressing model, where each address is a tuple of a
rank in the job (or global process ID) and an address in that process. The
lower 48 bits of the address hold a virtual address in the process. The top
bit is set to indicate that the reference is a 2D address (as opposed to
linear address). This leaves 15 bits for network endpoint ID, which limits our
scalability to $2^{15}$ endpoints. Any node-local data can be made accessible
to other cores in the system by wrapping the address and node ID into a 2D
global address. This address can then be accessed with a delegate operation
and even be buffered by other cores. The address is converted at the destination
into a canonical x86 address by replacing the upper bits with the
sign-extended upper bit of the virtual address. 2D addresses may refer to
memory allocated from a single processes' heap or from a task's stack.
Figure~\ref{fig:memory-structure} shows how 2D and linear addresses can refer
to other cores' memory.


\begin{figure}[htbp]
  \begin{center}
    \begin{minipage}{0.95\columnwidth}
	\small
  \begin{description}
    \item[Allocation in the global heap:] \hfill \\
      	\lstinline[style=grappa]{GlobalAddress<T> global_malloc<T>( size )} \hfill \\
      	\lstinline[style=grappa]{global_free( GlobalAddress<T> )}  \hfill
    \item[Delegate operations:] \hfill \\
      	\lstinline|T    delegate_read( GlobalAddress<T>)|  \hfill \\
      	\lstinline|Promise<T> delegate_read_async( GlobalAddress<T> )|  \hfill \\
      	\lstinline|void delegate_write( GlobalAddress<T>, T value)| \hfill \\
      	\lstinline|void delegate_write_async( GlobalAddress<T>, T value)| \hfill \\
      	\lstinline|bool delegate_cas( GlobalAddress<T>, T cmp, T set)| \hfill \\
      	\lstinline|T    delegate_fetch_inc( GlobalAddress<T>, T inc)| \hfill \\
      	\lstinline|void delegate_inc_async( GlobalAddress<T>, T inc)|  \hfill\\ 
%        \lstinline|void buffer_acquire( GlobalAddress<T>, local_buf, {RO,RW,WO})| \hfill \\
%        \lstinline|void buffer_release( GlobalAddress<T>, local_buf )| \hfill
    % Perform buffer operations to acquire/release global data.
    % Acquire copies all data to local node and returns a pointer.
    % For write acquire, release copies data back to global memory.
	\end{description}
      \caption{\label{fig:accessing-memory} \Grappa API for memory accesses.}     \end{minipage}
  \end{center}
\end{figure}

\begin{figure}[t]
\begin{center}
  \includegraphics[width=0.95\columnwidth]{figs/memory-structure}
\begin{minipage}{0.95\columnwidth}
  \caption{\label{fig:memory-structure} Global memory referencing in \Grappa}
\end{minipage}
\vspace{-3ex}
\end{center}
\end{figure}

\paragraph{Global memory access} 
Access to \Grappa's distributed shared memory is provided through  {\em
delegate} operations, which are short memory accesses performed at the memory
location's home node. When the data access pattern has
low-locality, it is more efficient to modify the data on its home core rather
than bringing a copy to the requesting core and returning it after
modification. Delegate operations~\cite{Nelson:hotpar11, delegated:oopsla11}
provide this capability. Applications can dispatch computation to be performed
on individual machine-word sized chunks of global memory to the memory system
itself. Delegates can execute arbitrary non-blocking code, so we use them to perform simple
\emph{read\/}/\emph{write\/} operations to global memory, as well as more complex \emph{read-modify-write\/} operations (e.g., \emph{fetch-and-add\/}). 

Delegate operations are \emph{always\/} executed at the home core of their
address. The remote operation may not perform any operations that could cause
a context switch; this ensures any modifications are atomic. We limit delegate
operations to operate on objects in the 2D address space or objects that fit
in a single block of the linear address space so they can be satisfied with a
single network request. Given these restrictions, we can ensure that delegate
operations for the same address from multiple requesters are always serialized
through a single core in the system, providing atomic semantics without using
actual atomic operations (and thus avoiding their typical high cost).

Delegate operations can be either {\em blocking} or {\em non-blocking}.
With blocking operations, the task issuing the delegate call blocks until
the delegate operation completes, which is necessary, for example, to ensure
that synchronization has finished before continuing. Synchronization in \Grappa
is implemented using blocking delegate operations. On the other hand, remote data
accesses often can overlap. In order to avoid
unnecessary waiting, we support non-blocking delegate operations. For reads,
we support a ``futures''-like mechanism. Tasks may issue delegate reads in
parallel and block on the ``promises'' returned by non-blocking delegate
invocations. Delegate write operations may also be performed as
non-blocking,
and since they do not return data, a mechanism to detect completion is needed.
\Grappa provides a \texttt{GlobalCompletionEvent} synchronization object, which non-blocking operations (including tasks) can be enrolled with and which other tasks can block on to be notified and woken when all enrolled operations in the collection are complete.
\TODO{augment this with our bulk futures ie global completions}

When programmers want to operate on data structures spread across
multiple nodes, accesses must be expressed as multiple delegate
operations along with with appropriate synchronization
operations. \Grappa's API also includes calls for gathering and scattering
contiguous blocks in the global heap, but the user is responsible for
ensuring correct synchronization.

\paragraph{Memory consistency model discussion} As mentioned earlier, all
synchronization operations are done via delegate operations. Since they all
execute on their home core in some serial order, they are guaranteed to be globally linearizable~\cite{herlihy1990linearizability}, with their
updates visible to all cores across the system in the same order. In addition, only one synchronous delegate will be in flight at a time from a particular task. Therefore, synchronization operations from a particular task are not subject to reordering. 
% \TODO{I think we can support multiple delegates in parallel from a task as
% long as we block on them before counting on them being complete. see my
% description above about the \emph{future/promise} delegates we have now. Not
% sure if there's a strong example of when this is useful for synchronization
% (though it's definitely useful for reads/writes), acquiring multiple locks
% doesn't work because we have to acquire locks in order to prevent deadlock.}
Consequently, all synchronization operations execute in program order and are
made visible in the same order to all cores in the system. These properties
are sufficient to guarantee a memory model that offers sequential consistency
for data-race-free programs~\cite{AdveHill1990} (all accesses to shared data
are separated by synchronization). This is the memory model that underpins
C/C++~\cite{N2480,N2800}.

Note, however, that if the application code uses explicit buffers or
asynchronous delegates to access shared data, all updates must be published back to
the home core before the synchronization operation that protects the data is
performed. This is done using release operations on cached regions and using
the \texttt{GlobalCompletionEvent} object to determine that asynchronous
delegates have completed.



\TODO{admit the lack of a synchronization feature for lightweight
    transactions across two or more domains. Talk about using the same
    philosophy of moving data along with the sync (same as a real
    cache) and utilizing latency tolerance. Then say further is out of
    scope for this paper.}



\section{Communication Support}
\label{sec:communication}

\Grappa's communication support has two layers: user-level messaging interface
based on active messages; and network-level transport that supports request
aggregation for better communication bandwidth.

% In order to mitigate the low message injection rate limits of commodity
% networks, Grappa's communication stack has two layers:

\paragraph{Active messages interface} At the upper (user-level) layer, \Grappa
implements asynchronous active messages~\cite{vonEicken92}. Each message
consists of a function pointer, an optional argument payload, and an optional
data payload. 

% When a task sends a message, the message is linked to a send queue
% associated with the message's destination node and the task continues
% execution. Note that the message contents stay in the worker's stack until
% it is sent to the network interface buffer, to avoid multiple copies of the
% payload. 

\paragraph{Message aggregation} In our experiments the vast majority of upper
layer message requests are smaller than 44 bytes. Our measurements confirm
manufacturers' published data [15]; with 44-byte packets, the available
bisection bandwidth is only a small fraction (3\%) of the peak bisection
bandwidth. As mentioned earlier, commodity networks including InfiniBand
achieves their peak bisection bandwidth only when the packet sizes are
relatively large --- on the order of multiple kilobytes. The reason for this
discrepancy is the combination of overheads associated with handling each
packet (in terms of bytes that form the actual packet, processing time at the
card and processing on the CPU within the driver stack). Consequently, to make
the best use of the network, we must convert small messages into large ones.

\paragraph{Message processing mechanics} Since communication is very frequent
in \Grappa, aggregating and sending messages efficiently is very important. To
achieve that, \Grappa makes careful use of caches, prefetching and lock-free
synchronization operations. \TODO{Does the explanation below need a figure?}

Outgoing messages are organized as an array of linked lists, with one linked
list per destination node in the system. Each core in a given node is
responsible for aggregating and sending the resulting message to a set of
destination nodes. The outgoing message lists are located in a region of
memory shared across all cores in a \Grappa node (thus enabling cores to peak
at each other's message lists). When a task sends a message, it allocates a buffer (typically on its stack),  determines the destination system
node, and links the buffer into the corresponding linked list.

Each core has a system worker per destination node it is responsible for; this
system worker periodically checks whether messages to the corresponding
destination node is either large enough or has waited past a time-out period,
at which point the message is sent out. Aggregating and sending a message
involves manipulating a shared data-structure (the message list). This is done
using CAS (compare-and-swap) operations to avoid high synchronization costs. 

% Threads that dispatch a message allocate a message buffer (typically on their
% stack) and dispatch a send request. \Grappa examines the destination system
% node for this field, and adds the message to a linked list of messages
% destined for that node. Periodically, or when this linked list grows large,
% the list is transferred using a CAS (compare-and-swap} operation to the
% processor on the originating system node that is responsible for managing all
% outbound communication requests to the specified destination node. This
% two-step process is used because it was determined empirically that we could
% only infrequently use processor coherent access requests because they remain a
% bottleneck on modern multicore hardware.

Each node has a region of memory with send buffers where the final aggregated
messages are built. These buffers are visible to the network card, and
messages can be sent with user-mode operations only. When the worker
responsible for outbound messages to a given system node has received a
sufficient number of message send requests, or a timeout is reached, the
linked list of messages is walked and messages are copied to a send buffer.
This process requires careful prefetching because most of the outbound
messages are \emph{not} in the processor cache at this time (recall that a
core can be aggregating messages originating from other cores in the same
node). Once the send buffer has been formed it is handed off to GASnet for
transfer to the remote system node. RDMA is used if the underlying network
supports it. 

There are two useful consequences of forming the send buffer at the time of
message transmission instead of along the way, as individual upper layer
message send requests are received. First, as previously mentioned most of the
messages are not in the cache and prefetching is used to run-ahead in the
linked list of messages in order to avoid cache misses. But once the send
buffer is formed it is in the cache (for the most part). Hence, when it is
handed off to GASnet for transfer across the physical wire, the network card
can pull the message buffer from the processor cache instead of main memory,
which we have found speeds performance. The second consequence of this
decision is that we do not need to pre-allocate buffers for all destination
nodes in the system, as the buffer can be allocated on the fly. Nevertheless
we have found it efficient to build a flow-control like protocol of
outstanding message buffers between pairs of system nodes.

Once the remote system node has received the message buffer it is unpacked.
Upper level \Grappa messages are delivered to each core at the destination
system node by examining the message buffer at each processor core and then
handing the message buffer to the next core for delivery of messages destined
to that core.



%There are three situations in which a queue of aggregated messages is sent:
%(1) Each queue has a message size threshold of 4096 bytes, chosen to give
%reasonable network performance. If the size in bytes of a queue is above the
%threshold, the contents of the queue are sent immediately. (2) Each queue
%has a wait time threshold ($\approx${1ms}). If the oldest message in a queue
%has been waiting longer than this threshold, the contents of the queue are
%sent immediately, even if the queue size is lower than the message size
%threshold. (3) Queues may be explicitly flushed in situations where the
%programmer wants to minimize the latency of a message at the cost of bandwidth
%utilization.

%The network layer is serviced by polling. Periodically when a context
%switch occurs, the Grappa scheduler switches to the network polling
%thread. This thread has three responsibilities. First, it polls the
%lower-level network layer to ensure it makes progress. Second, it
%deaggregates received messages and executes active message
%handlers. Third, it checks to see if any aggregation queues have
%messages that have been waiting longer than the threshold; if so, it
%sends them.

%Underneath the aggregation layer, Grappa uses the \gasnet~communication
%library~\cite{gasnet} to actually move data. All interprocess
%communication, whether on or off a cluster node, is handled by the
%\gasnet~library. \gasnet~is able to take advantage of many communication
%mechanisms, including ethernet and infiniband between nodes, as well as
%shared memory within a node.

%Some networks provide access to a remote machine's memory directly. This
%would seem to be a good fit for a programming model focused on global
%shared memory, but in fact we do not use it. In our experiments, we
%found that RDMA operations are subject to the same message rate
%limitations as all other messages on these cards, and thus using raw
%RDMA operations for our small messages would make inefficient use of
%bandwidth. Instead, we implement remote memory operations with active
%messages. A byproduct of this design decision is that Grappa is not
%limited to RDMA-capable networks.

%\TODO{We should expand on the zero-copy aggregation stuff?}

% there is a message pointer list in each core. this list has as many entries
% as the number of cores in the system. each entry is another list of message
% with that destination.


% each core in a node is responsible for a set of nodes in the system

% cores in a node share a region of memory

% network buffers also reside in this shared region of memory

% messages ptrs also contain prefetching information to speed up assembling message in the buffer.

% there are as many sending workers in a core as the number of nodes it is responsible for.






%\TODO{paper.tex: I know we removed the programming model from the paper and
    % merged into the overview pieces, but I think we need at least a
    % brief summary of the machine model the programmer should be
    % thinking about. 1.large number of tasks, 2.PGAS model for thinking
    % of memory accesses where remote memory is expensive and
    % packets should be large when this is easily expressed, 3.
    % streaming writes are cheaper than writes that require consistency
    % in the task, 4.overlapping read/writes are better than
    % nonoverlapping. Thats a lot to think about but a compiler can at
    % least help with part of 2 and 4 and \Grappa lets you work less
    % hard on 2 by providing aggregation. -BM}

\section{Methodolody} \label{sec:method}

To explore the performance of the Grappa runtime we have implemented three algorithms: breadth first search, betweenness centrality, and unbalanced tree search.  These algorithms were implemented for the Cray XMT (our baseline) and for Grappa.  Performance results for Grappa were obtained on a 144 node cluster.  Individual nodes of this cluster contain the hardware depicted in Table~\ref{table:grappanode}.  Nodes are interconnected with both 10G Ethernet and Infiniband.  For all but startup and configuration, they are configured to communicate over the Infiniband network for these experiments.

The metric we use is algorithmic time, which means startup and loading of the data structure (from disk) is not included in the measurement.  Data is collected on real systems, which means minor variations in runtime exist from run to run.  The average of multiple runs are used, and where appropriate, confidence in the result is reported (error bars, standard deviation, etc).

One question that must be answered when making a comparison between
the Cray XMT and a typical HPC cluster is what is a fair comparison --
these systems are quite different.  Three options immediately come to
mind: equal number of processing cores, equal number of network
interfaces, and equal dollars.  \checkme{We discounted the last option fairly
quickly, because it isn't a lasting data-point and, as will be evident
later, would be hugely unfair to the XMT.}  The first option, cores,
has some merit, but Grappa is designed for applications that have no
locality in their computation.  This means almost all of their memory
accesses are remote.  Hence, the factor that limits their performance
is not processing, but communicating.  Hence, we have chosen the
middle option, network interfaces as the way to normalize across the
XMT and our cluster.  Each processor in the XMT system has its own
network interface to access shared memory.  In the HPC cluster we use,
each processor (which contains up to 32 cores), has a single Infiniband interface.  Hence, for our results we scale up XMT processors one for one with full system nodes.



\subsection{Systems}
For measurements, we run Grappa on a 144-node cluster of AMD
Interlagos processors. Nodes have 32-core (every pair share a floating-point
unit) 2.1-GHz processors, 64GB of memory, and \TODO{xxx 40Gb network card}. \TODO{cluster
    uses xxx Infiniband switch}. We configure the nodes to have 32 1-GB
hugepages to minimize TLB misses for the random access patterns we
expect from irregular applications. The results we present are for
this machine, but we have also run Grappa on our own 12-node Intel
Xeon Westmere cluster.

We compare to the MTA using a 128-node Cray XMT (3rd generation MTA). 
Each node consists of a 500-MHz MTA Threadstorm multithreaded
processor that supports 128 streams. The machine uses Cray's proprietary
SeaStar2 interconnection network.

\subsection{Applications}
\TODO{
    Traversal of an unbalanced, unpartitioned tree captures the foundations for implementing most forms of irregular parallelism. 
Recursion is found in every irregular divide and conquer algorithm.
Loops, even regular ones, must be dynamically scheduled to get good
load balance when run on a large, non-uniform system. Nested
parallelism involves fine-grain creation of dynamic amounts of work.
Dataflow..
}

\paragraph{Unbalanced tree search in-memory (UTS-Mem)}
Unbalanced Tree Search (UTS) is a benchmark
for evaluating the programmability and performance of systems for
parallel applications that require dynamic load balancing
\cite{Olivier:uts2006}. It involves traversing an unbalanced implicit
tree: at each vertex, its number of children is sampled from some
probability distribution, and this number of new nodes are added to a
work queue to be visited. While this benchmark captures irregular,
dynamic parallelism, we want to evaluate performance on algorithms
with irregular memory access patterns.

We augment UTS by using the existing traversal code to
create a large tree in memory, and then we traverse the in-memory
tree. We call this benchmark UTS-Mem, and the timed portion is this
traversal of the in-memory tree. This in-memory traversal is
unknowledgeable of the tree structure beforehand.

%todo: say tree explore is same as a search except we are additionally needing to synchronize on all vertices visited

The Grappa version of the in-memory tree search uses the
asynchronous parallel for loop over a visited vertex's children 
list. A larger threshold can be set to take advantage of this 
locality by processing more child pointers sequentially in a single task.

\comment{The Grappa version of tree search UTS-mem

\lstset{language=C++,
       basicstyle=\footnotesize,
       tabsize=2}
\begin{lstlisting}
// base of shared vertex_t[]
GlobalAddress<vertex_t> Vertices;        

// base of shared int64_t[]
GlobalAddress<int64_t> ChildrenPointers;  

void search_vertex( id ) {
  GlobalAddress<vertex_t> v_addr = Vertices + id;
  Incoherent<vertex_t>::RO v( v_add, 1 );

  // start index of my children pointers
  childIndex = v->childIndex;    

  // how many children I have
  numChildren = v->numChildren;  

  // parallel loop over the child list for this vertex
  parallel_for(fn=&search_children, start=childIndex, iters=numChildren);
}

void search_children( start, iters ) {
  // take advantage of spatial locality in the array of children
  GlobalAddress<int64_t> child_base_addr = ChildrenPointers + start;
  Incoherent<int64_t>::RO childIds( child_base_addr, iters );

  // spawn a task to visit each child
  for (i = 0..iters) {
    SoftXMT_publicTask( fn=&search_vertex, id=childIds[i] );
  }
}
\end{lstlisting}
}

\paragraph{Breadth-first-search (BFS)} This is the primary kernel for the Graph500 benchmark and is what currently determines the ranking of machines on the Graph500 list~\cite{graph500list}. As a whole, the Graph500 benchmark suite is designed to bring the focus of system design on data-intensive workloads, particularly large-scale graph analysis problems, that are important among cybersecurity, informatics, and network understanding workloads. The BFS benchmark builds a search tree containing parent nodes for each traversed vertex during the search.  While this is a relatively simple problem to solve, it exercises the random-access and fine-grained synchronization capabilities of a system as well as being a primitive in many other graph algorithms. Performance is measured in \emph{traversed edges per second} (TEPS), where the number of edges is the edges making up the generated BFS tree. One of the reference implementations of Graph500 BFS is for the XMT. We compare this against a straightforward Grappa implementation that algorithmically matches the XMT reference code to allow us to fairly compare the execution of the two systems. We do not employ algorithmic improvements beyond the reference code, though there are many \cite{Beamer:Graph500,Yoo:FixedPointGraph500}.




\section{Evaluation}
To evaluate the impact flat combining has on the global data structures, we ran a series of experiments to test the raw performance of the data structures themselves under different workloads, and further measured their impact on performance of two simple graph benchmarks.

Experiments were run on a cluster of AMD Interlagos processors. Nodes have 32 2.1-GHz cores in two sockets, 64GB of memory, and 40Gb Mellanox ConnectX-2 InfiniBand network cards, connected via a QLogic InfiniBand switch.

\subsection{Data Structure Throughput}
First we measured the raw performance of the global data structures on extremely simple throughput workloads.

% \begin{figure*}[t]
%   \centering
\begin{figure}[t] %{0.5\textwidth}
  \centering
  \includegraphics[width=0.5\textwidth]{data/plots/vector_perf.pdf}
  \caption{\emph{GlobalStack and GlobalQueue.}
    Results are shown on a log scale for a throughput workload performing 256 million operations with 2048 workers per core and 16 cores per node. Flat combining improves throughput by at least an order of magnitude and allows performance to scale. Matching pushes and pops enables the stack to perform even better on a mixed workload.
  }
  \label{fig:vector}
\end{figure}
  % \hspace{0.05\textwidth}
  % % ~ %add desired spacing between images, e. g. ~, \quad, \qquad etc.
  % %(or a blank line to force the subfigure onto a new line)
  % \begin{subfigure}[b]{0.45\textwidth}
  % \centering
  % \includegraphics[width=\textwidth]{data/plots/queue_perf.pdf}
  % \caption{\emph{GlobalQueue.} Same parameters as stack performance results. The queue is unable to do matching locally, but benefits from reducing the amount of synchronization that must globally serialize. The mixed workload performs worse because the current implementation serializes combined enqueue and dequeue operations.}
  % \label{fig:queue}
  % \end{subfigure}
  % \hspace{0.05\textwidth}
  
  %  
\begin{figure}[t]
  \centering
  \includegraphics[width=0.5\textwidth]{data/plots/hash_perf.pdf}
  \caption{\emph{GlobalHashSet and GlobalHashMap.}
    Results are shown for a throughput workload inserting and looking up 256 million random keys in the range 0-1024 into a global hash with 1024 cells, with 2048 workers per core and 16 cores per node.
    Performance without combining scales out to 32 nodes because synchronization happens at each hash cell, but drops off as the number of destinations increases. Due to eliminating duplicate inserts and lookups, the combining version is able to continue to scale.}
    \TODO{if time: add 'delete' operation, too (show that it doesn't affect correctness or performance)}
  \label{fig:hash_perf}
\end{figure}
  % %
  % \hspace{0.05\textwidth}
  % %
  % \begin{subfigure}[b]{0.45\textwidth}
  % \centering
  % \includegraphics[width=\textwidth]{data/plots/hashmap_perf.pdf}
  % \caption{\emph{GlobalHashMap.} Same workload as for the Set, but random integers as the values in the map. Performance matches that of the Set.} \TODO{try increasing size of value "payload", currently is just a tiny int64.}
  % \label{fig:hashmap}
  % \end{subfigure}%
  %
%   \caption{
%     \emph{Raw performance of global data structures on simple throughput workloads.}
%     Results shown for workloads with varying mixtures of writing and reading operations. All experiments were performed with 16 cores per node, and a suitable set of Grappa runtime parameters fixed for a given plot.
%     \TODO{either all with error bars or none?}
%     \TODO{combine stack/queue \& set/map plots, no point in saying the same things twice.}
%   }\label{fig:datastructs}
% \end{figure*}

\paragraph{Queue and Stack}
The GlobalQueue and GlobalStack have very similar implementations in terms of how they are synchronized. Even with Grappa's aggregation, without combining, both the stack and queue completely fail to scale because of serialization of all workers' updates on the master core.
With combining, both scale well out to 64 nodes.
On the mixed workload, the stack is able to do matching locally, allowing it to reduce the amount of communication drastically, greatly improving its performance.

The queue also benefits from reduced synchronization and batching operations, and its all-push workload performs identically to the stack's.
However, the queue is unable to do matching locally, and in fact, the mixed workload performs worse because the current implementation serializes combined enqueue and dequeue operations. This restriction could be lifted with more careful synchronization at the master core allowing enqueues and dequeues to proceed in parallel as long as they do not conflict.

\paragraph{HashSet and HashMap}
The GlobalHashSet and GlobalHashMap have the same synchronization strategy (serialization happens at each hashed location) which allows them to scale fairly well even without combining because serialization only happens on conflicts. However, after 32 nodes, scaling drops off significantly due to increased number of concurrent accessors and more destinations.
The combining version is able to perform repeated inserts and repeated lookups with a single remote operation, enabling it to continue scaling further.

\subsection{Application Kernel Performance}
The original goal was to have scalable global data structures in the Grappa library that could be used in applications. Naive implementations of the data structures are insufficient for scaling, so HPC applications often implement their own highly customized versions. However, this forces each application to carefully reason about the ways in which the structures will be accessed. Additionally, many optimizations rely on relaxing global consistency in a way that does not affect the program. 
The Grappa data structures are synchronized to provide the most general use and match the expectations of programmers and algorithms.

\paragraph{Breadth-First Search}
\begin{figure}[t]
  \centering
  \includegraphics[width=0.5\textwidth]{data/plots/bfs_perf.pdf}
  \caption{\emph{BFS} on a Graph500-spec graph of scale 26 (64 million vertices, 1 billion edges), with the direction-optimizing BFS algorithm. Performance is measured in millions of Traversed Edges Per Second (MTEPS).}
  \label{fig:bfs_perf}
\end{figure}
The first application kernel is the Graph500 Breadth-First-Search (BFS) benchmark. This benchmark does a search starting from a random vertex in a synthetic graph and builds a search tree of parent vertices for each vertex traversed during the search. While this is a relatively simple problem, it exercises the random-access throughput of a system as well as being a primitive in many other graph algorithms. The BFS algorithm contains a global queue which represents the frontier of vertices to be visited in each level.
Our implementation employs the direction-optimizing algorithm by Beamer et al.\cite{Beamer:Graph500} which performs particularly well for the scale-free RMAT graphs generated by the benchmark.
The frontier queue in BFS is amenable to further optimization to take advantage of the fact that the algorithm does pushes and pops in separate phases, allowing consistency to be relaxed.
We compare our implementation of BFS using the flat-combined global queue described above with a highly tuned Grappa implementation that uses a custom asynchronous queue.

Figure~\ref{fig:bfs_perf} shows the results of scaling the BFS kernel up to 64 nodes. The simple queue implementation without flat combining is completely unscalable. However, the flat combining queue tracks the highly tuned asynchronous version. This illustrates that providing a safe, synchronized data structure for initially developing algorithms for PGAS is possible, while further optimizations can be applied incrementally.

\paragraph{Connected Components}
\begin{figure}[t]
  \centering
  \includegraphics[width=0.5\textwidth]{data/plots/cc_perf.pdf}
  \caption{\emph{Connected Components} on the same scale 26 Graph500 graph. Performance is measured in MTEPS.}
  \label{fig:cc_perf}
\end{figure}
Connected Components (CC) is another simple graph analysis kernel that illustrates another use of global data structures in irregular applications. We implement the three-phase CC algorithm~\cite{mtgl} which was designed for the massively-parallel MTA-2 machine. In first phase, the algorithm begins many traversals in parallel from random starting vertices, labeling vertices with the root vertex. Whenever two traversals encounter each other, their searches are pruned and an edge between the two roots is inserted in a set. After all edges have been traversed in this way, the set of edges forms a new, typically much smaller, graph. The second phase performs the classical Shiloach-Vishkin parallel algorithm\cite{shiloach1982n} on this reduced graph, and the final phase propagates the component labels out to the full graph.
The creation of the reduced-graph edge set dominates the runtime of this algorithm, so improving the implementation of the set operations has a significant impact on performance. As in the case of BFS, further optimizations involving relaxation of consistency can be applied, in this case, to the global set. Therefore, we compare our straightforward implementation using the generic GlobalHashSet with and without flat combining against a tuned asynchronous implementation.

The results in Figure~\ref{fig::cc_perf} show that none of these three implementations scale very well out to 64 nodes. However, performing combining does improves the performance of the algorithm overall and improves scaling. The tuned version outperforms the synchronous version because it is able to build up most of the set locally on each core before merging them at the end of the first phase. An implementation of the GlobalHashSet that did not provide synchronized semantics could potentially relax consistency in a more general way, but this is left for future work.


\section{Background}

% The goal of the Grappa framework is to simplify the task of implementing
% irregular applications.
Grappa is built on many existing ideas in programming languages, systems and
architecture. In this section we discuss related frameworks and key enabling
technologies that Grappa builds upon.

\paragraph{Comparable frameworks} Distributed graph processing frameworks like
Pregel \cite{pregel:2010} and Distributed GraphLab \cite{distgraphlab:vldb12}
share similar goals as Grappa. Pregel adopts a bulk-synchronous parallel (BSP)
execution model, which makes it inefficient on workloads that could prioritize
vertices. GraphLab, on the other hand, schedules vertex computations
individually, allowing prioritization, which gives faster convergence in a
variety of iterative algorithms.\TODO{GraphLab is not general. Myers will
expand on that here. Make a strong point about generality.} Grappa also
supports dynamic parallelism with asynchronous execution, but parallelism is
expressed as tasks or loop iterations.\TODO{Should RAMcloud go here?}.

\paragraph{Global memory} Grappa includes a custom implementation of a
software distributed shared memory (DSM) system. Many traditional software DSM
systems are page based~\cite{Treadmarks,munin} and aim to hide the fact that
they are built in software from applications by exploiting the processor's
paging mechanisms, therefore relying heavily on locality. Instead, Grappa,
like other partitioned global address space (PGAS) models, implements its DSM
at the language, rather than system level. Languages such as Chapel
\cite{Chamberlain:2007}, X10 \cite{X10:2005}, and UPC \cite{upc:2005} make accesses to shared
structures look like normal memory references. As we describe later, Grappa
chooses a middle ground, where global addresses are explicit in the API and
local accesses are emitted conventionally by the compiler. Similar to DSM
systems, Grappa provides a consistency model that ensure high performance and
with good programmability properties. % Caching is Cache operations are explicit in the  exposed to in the
% API, however, allowing software to optimize their usage.

We decided to build a custom distributed shared memory implementation, as
opposed to using an existing implementation, for two key reasons: (1) in order
to support the large number of tasks required for latency tolerance, the
implementation of the global memory system is tightly coupled to the task
scheduler; and (2) since commodity networks only do efficient RDMA or ordinary
transmission of small messages, the implementation of the global memory system
must make use of the network message aggregator.

% 

\paragraph{Multithreading}
Grappa uses multithreading to tolerate memory latency. This is a well known
technique. Hardware implementations include the Tera MTA \cite{tera:mta1}, Cray XMT
\cite{feo:xmt}, Simultaneous multithreading \cite{tullsen:smt}, MIT Alewife
\cite{agarwal:alewife}, Cyclops \cite{almasi:cyclops}, and even GPUs \cite{gpus}. As we
describe in this paper, Grappa use a lightweight user-mode task scheduler to
multiplex thousands of tasks on a single processing core. The large number of
tasks is required because of the extremely large internode latency.


\section{Conclusion}

Irregular computations are both important and challenging to execute quickly.
Scaling these applications easily to commodity hardware has been a historical
challenge. \Grappa is a runtime framework that simplifies this task for
software developers and compiler writers. This paper describes the \Grappa
framework and its three main components: a task library, a distributed shared
memory system, and a network aggregator to make commodity networks efficient
with small message sizes. \Grappa's key aspect is extreme latency tolerance,
which not only hides network latency but also enables the system to spend time
on sophisticated work stealing and network optimizations, trading latency for
even more throughput.

Our evaluation of \Grappa reveals that the core components, scheduling and communication, achieve their design goals.  Thousands of workers can be efficiently context switched on a multicore processor, up to the DRAM bandwidth.  Aggregating messages enables \Grappa to achieve over 1.0 GUPS on 64 nodes.  We also explored four other algorithms: unbalanced tree search, breadth first search, PageRank, and integer sort, comparing performance of these algorithms on \Grappa, the Cray XMT, and against hand-optimized MPI implementations.  As would be expected, when the MPI implementation exploits in the application, many of the same techniques \Grappa provides in the runtime, such as aggregating requests, yet without the overhead of tasking, the MPI implementation excels.  This comes with enormous implementation complexity burdened on the application developer, however, and significant tuning has yet to be done on the \Grappa runtime system.  Compared to the Cray XMT \Grappa is between 2X faster and 4X slower.  Yet when cost-performance is considered, mass-market x86 cluster hardware makes \Grappa a highly attractive option.


%
\section{Acknowledgements}

A portion of this work was performed using PNNL Institutional
Computing at Pacific Northwest National Laboratory; we thank Tim
Carlson for his help in configuring the cluster. We thank the HPC
Advisory Council for the use of their clusters.


%\subsection{References}
\bibliographystyle{abbrv}
\bibliography{paper,generals}

\end{document}

