\section{Related Work}

\vspace{0.5ex}
\noindent{\bf Multithreading} 
%\section{Multithreading}
\Grappa uses multithreading to tolerate memory
latency. This is a well known technique. Hardware implementations include the
Denelcor HEP~\cite{hep}, Tera MTA~\cite{tera:mta1}, Cray XMT~\cite{feo:xmt}, Simultaneous
multithreading~\cite{tullsen:smt}, MIT Alewife~\cite{agarwal:alewife},
Cyclops~\cite{almasi:cyclops}, and even GPUs~\cite{gpus}.

\Grappa's closest ancestor is the Threaded Abstract
Machine~\cite{CullerGSvE93}. This was a software runtime system
designed as a prototyping platform for dataflow execution models on
distributed memory supercomputers, and contained support for
inter-node communication and management of the memory hierarchy as
well as context switching and scheduling computation. The Active
Messages~\cite{vonEicken92} work that grew out of
this project inspired our communication layer. One of the conclusions
of this work~\cite{Culler:1993:TFL:647025.714362} was that context
switch costs can be low only when contexts are in cache, and that
latency tolerance was not sufficient to guarantee performance on
commodity processors. \Grappa demonstrates that times have changed:
modern commodity processors have sufficient bandwidth and prefetch
capacity to stream contexts from DRAM and sustain a very large number of active contexts. 


\Grappa implements its own software-based multithreading with a
lightweight user-mode task scheduler to multiplex \emph{thousands\/} of tasks
on a single processing core. The large number of tasks is required to tolerate
the very high inter-node communication latency of commodity networks.
\Grappa's task library employs several optimizations: an extremely fast task
switch, a small task size, and judicious use of software prefetch instructions
to bring task state into the cache sufficiently long before that task is
actually scheduled. The main difference between \Grappa's support for
lightweight threads and prior work such as
QThreads~\cite{Wheeler08qthreads:an} and
Capriccio~\cite{Behren03capriccio:scalable} is context prefetching, which is needed for good performance when multiplexing such a large
number of tasks.

\vspace{0.5ex}
\noindent{\bf Software distributed shared memory.} 
%\section{Software distributed shared memory} 
The goal of providing a shared memory abstraction for a distributed
memory system goes back nearly 30 years. Much of the innovation in SDSM has occurred around reducing the
synchronization cost of doing updates. The first DSM systems,
including IVY~\cite{Li:1989:MCS:75104.75105}, used frequent invalidations to
provide sequential consistency, which imposed significant
communication cost for write-heavy workloads.  Later systems supported
more relaxed consistency models, including release consistency, which
reduced the cost by allowing updates to be buffered until
synchronization occurred. Furthermore, multiple writer protocols that
sent only modified data were developed to help combat false
sharing. The
Munin~\cite{Bennett:1990:MDS:99163.99182,Carter:1991:IPM:121132.121159}
and TreadMarks~\cite{Keleher:1994:TDS:1267074.1267084} systems
exploited both of these ideas, but some coherence overhead was still
required. In contrast, \Grappa's delegate-based approach to updates
avoids synchronization overhead entirely, providing sequential
consistency for data-race-free programs without the cost of a
full coherence protocol. 
% This cost of this communication is mitigated through latency
% tolerance.


Another way in which SDSM systems differ is in the granularity of
access control. Many SDSM systems, including IVY and TreadMarks,
tracked ownership at a granularity of a kilobyte or more. There are
two main justifications for this design choice: first, networks are
more efficient with large packets, and second, multi-kilobyte
granularity allowed systems to reuse the processor's paging mechanisms
to accelerate access control checks and to provide shared memory
transparently. Unfortunately, this meant that these systems depended
on a fair ammount of locality to amortize the cost of moving these large
blocks, and the systems were very susceptible to false sharing. Other
systems, including Munin and
Blizzard~\cite{Schoinas:1994:FAC:195473.195575}, allowed tracking
ownership with variable granularity to address these problems.
\Grappa's delegate-based approach is similar; since updates are always
performed at data's home node, only modified data needs to be moved.


SDSM systems often required extensive modifications to the system
software stack, including the OS (IVY, Blizzard) and compilation
infrastructure (Munin). This made these systems difficult to port to
new platforms. \Grappa follows the lead of TreadMarks and provides
SDSM entirely at user-level through a library and runtime.

In summary, while \Grappa's DSM system is conceptually similar to prior work,
we accept the random access aspect of irregular applications and optimize for
throughput rather than low latency. Our DSM system must support enough 
concurrency to tolerate the latency of the network and additional latency
overhead impose by the runtime system.

\vspace{0.5ex}
\noindent{\bf Partitioned Global Address Space languages.} 
%\section{Partitioned Global Address Space languages} 
The high-performance computing community has largely discarded the
coherent distributed shared memory approach in favor of the
Partitioned Global Address Space (PGAS)
model. Example include Split-C~\cite{Krishnamurthy:1993:PPS:169627.169724},
Chapel~\cite{Chamberlain:2007}, X10~\cite{X10:2005},
Co-array Fortran~\cite{Numrich:1998:CFP:289918.289920} and
UPC~\cite{upc:2005}. \Grappa shares many parts of its design philosophy
with these languages.

There are two key ideas \Grappa draws from PGAS languages. First, PGAS
languages implement DSM at the language, rather than the system
level. This allows for a number of optimizations, including efficient
split-phase access, variable data granularity, and support for
user-customizable synchronization operations. \Grappa follows this
approach for the same reason.

Second, in PGAS languages, each piece of data has a single canonical location
on a particular node. \Grappa takes the same approach. However, PGAS languages
often expect programmers to modify algorithms to take advantage of locality by
processing node-local data as much as possible; access to data stored on other
nodes is possible but is seen as something to avoid if possible. \Grappa takes
the opposite view, as it optimizes for random access to data anywhere in the
cluster, and exploiting any available locality in the application is a
secondary concern.

Most PGAS languages adopt a SPMD programming model: the programmer
must reason about what is happening on every node in the
system. Chapel is the exception: it provides a global view of control,
while allowing programmers to direct individual cores when
necessary. \Grappa follows the same approach as Chapel, providing a
single-machine abstraction to the programmer along with support for
controlling the locality of computation when requested.


\vspace{0.5ex}
\noindent{\bf Distributed graph processing systems.} 
%\section{Distributed graph processing systems} 
While \Grappa is a general runtime system for any large-scale
concurrent application, it has been designed to perform especially
well on graph analysis. Other distributed graph processing frameworks
include Pregel~\cite{pregel:2010} and Distributed
GraphLab~\cite{distgraphlab:vldb12}. Pregel adopts a bulk-synchronous
parallel (BSP) execution model, which makes it inefficient on
workloads that could prioritize vertices. GraphLab overcomes this
limitation with an execution model that schedules vertex computations
individually, allowing prioritization, which gives faster convergence
in a variety of iterative algorithms. GraphLab, however, imposes a
rigid model that requires programmers to express computation as
transformations on a vertex and its edge list, with information only
from adjacent vertexes. Pregel is only slightly less restrictive, as
the input data can be any vertex in the graph. \Grappa also supports
dynamic parallelism with asynchronous execution, but parallelism is
expressed as tasks or loop iterations, which is a far more general
programming model for irregular computation tasks.
PowerGraph~\cite{powergraph:osdi12} improves the performance of
GraphLab for real-world graphs with power-law degree distributions by
using a vertex cut for graph partitioning. Algorithmic transformations
like this  and would also improve the
performance of graph applications on \Grappa.

While the bulk-synchronous MapReduce~\cite{Dean:2008:MSD:1327452.1327492} model and
related systems such as Hadoop~\cite{hadoop} are not a good fit for
irregular or graph problems, the ideas have been extended to support a
subset of related problems that require iteration, including some
graph-based machine learning
applications. HaLoop~\cite{Bu:2010:HEI:1920841.1920881} and
Spark~\cite{Zaharia:2010:SCC:1863103.1863113} are two examples of this
sort of system. This class of systems is generally characterized by
having a restricted programming model and restricted interface to
access data, focused on streaming data from disk. In particular,
direct communication between nodes in the cluster is not supported;
communications must happen through the  parallel data
structures provided. \Grappa is general enough to run these jobs, but the
MapReduce-derived systems have much more complete IO support and the
ability to tolerate from node failures.


